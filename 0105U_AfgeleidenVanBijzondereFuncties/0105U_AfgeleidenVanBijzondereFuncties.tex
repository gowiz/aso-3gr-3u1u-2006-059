\documentclass[a4paper,12pt]{article}

\input{../gowiz.tex}

\usepackage{versions}
\excludeversion{theorie}

\begin{document}

\pagestyle{fancy}
\lhead{}
\rhead{Oefeningen Afgeleiden van bijzondere functies}

\begin{theorie}

\thispagestyle{empty}
\begin{center}
  \begin{mdframed}
  \centering
  \fontsize{40}{50}\selectfont Afgeleiden van bijzondere functies
  \end{mdframed}
  \vfill
  \vfill
\end{center}
\subsection*{Doelstelling}
Je kan de afgeleiden functie bepalen van\hfill  {\scriptsize(LP 2006-059, LI 1.6.13, 1.7.8, 1.8.6, 1.9.6)}
\begin{itemize}
  \item goniometrische functies
  \item exponentiële functies
  \item logaritmische functies
  \item irrationale functies
\end{itemize}


\pagestyle{empty}
\mbox{}
\newpage
\clearpage
\thispagestyle{empty}
%\mbox{}
\tableofcontents
\newpage
\clearpage
\pagenumbering{arabic}

\pagestyle{fancy}
\lhead{}
\rhead{Afgeleiden van bijzondere functies}

\end{theorie}

\onehalfspacing

\section{Afgeleide functie van een irrationale functie}

\begin{oefening}
Bereken
\begin{multicols}{2}
\begin{enumerate}[(a)]
  \itemsep0.5em
  \item $\displaystyle\left(2\cdot\sqrt{x}-1\right)'$
  \item $\displaystyle\left(\sqrt{3x+2}\right)'$
  \item $\displaystyle\left(\sqrt{2-4x}\right)'$
  \item $\displaystyle\left(\sqrt{x^2}\right)'$
  \item $\displaystyle\left(\sqrt{5x^2+2x-1}\right)'$
  \item $\displaystyle\left(\sqrt[3]{x}\right)'$
  \item $\displaystyle\left(\sqrt{x}+\dfrac{1}{\sqrt{x}}\right)'$
  \item $\displaystyle\left(\dfrac{1}{\sqrt{x^2+3}}\right)'$
  \item $\displaystyle\left(\sqrt[3]{\left(x+6\right)^2}\right)'$
  \item $\displaystyle\left(\sqrt[3]{x^2+6}\cdot\sqrt{x^2+6}\right)'$
  \item $\displaystyle\left(\dfrac{\sqrt[3]{x^2+2x+1}}{\sqrt[4]{x^2+2x+1}}\right)'$
  \item $\displaystyle\left(\sqrt[3]{2x-3}\cdot\sqrt{8x-12}\right)'$
  \item $\displaystyle\left(\dfrac{\sqrt{x+6}}{\sqrt[3]{x+2}}\right)'$
\end{enumerate}
\end{multicols}
\end{oefening}

\begin{oefening}
Bepaal de afgeleide functie $f'$ van\\
\begin{enumerate}[(a)]
  \itemsep0.5em
  \item $\displaystyle f(x)=4\cdot\sqrt{\frac{1}{2}x}$
  \item $\displaystyle f(x)=\sqrt{\frac{1}{2}x^3+\frac{1}{3}x^2+\frac{1}{4}x+\frac{1}{5}}$
  \item $\displaystyle f(x)=\sqrt{\dfrac{x^3+5x}{x^2-1}}$
  \item $\displaystyle f(x)=\sqrt[3]{\left(x^3+x^2+x+1\right)^4}$
  \item $\displaystyle f(x)=\sqrt[5]{\left(x^3+1\right)\left(x^2+1\right)}$
\end{enumerate}
\end{oefening}

\begin{oefening}
{\em \scriptsize Ijkingsproef industrieel ingenieur, ann de bodt, tanja van hecke}\\
Bereken de afgeleide van $u$ naar $x$ indien $u=r^{3/2}$ en $r=\sqrt{4+x^2}$.
\begin{enumerate}[(a)]
  \itemsep.3em
  \item $\dfrac{3}{2}\dfrac{x}{\sqrt{r}}$
  \item $\dfrac{3}{2}\sqrt{r}$
  \item $\dfrac{2}{5}r^\frac{5}{2}$
  \item $\dfrac{2}{5}xr^\frac{3}{2}$
\end{enumerate}
\end{oefening}

\pagebreak
\section{Afgeleide functie van een logaritmische functie}

\begin{oefening}
Bepaal de afgeleide functie $f'$ van
\begin{multicols}{2}
\begin{enumerate}[(a)]
  \itemsep0.5em
  \item $\displaystyle f(x)=\log_5 x$
  \item $\displaystyle f(x)=\log_{3e} x$
  \item $\displaystyle f(x)=\ln(x-6)$
  \item $\displaystyle f(x)=\log(5x^2+4x+3)$
  \item $\displaystyle f(x)=\log^2 x$
  \item $\displaystyle f(x)=\dfrac{1}{\log x}$
  \item $\displaystyle f(x)=\log \sqrt[3]{\left(4x^2+2x+1\right)^2}$
  \item $\displaystyle f(x)=\sqrt[3]{\log\left(4x^2+2x+1\right)^2}$
  \item $\displaystyle f(x)=\sqrt[3]{\log^2(4x^2+2x+1)}$
  \item $\displaystyle f(x)=\log x^2 - \log x$
  \item $\displaystyle f(x)=\log 3x + \log x + \log x$
  \item $\displaystyle f(x)=\dfrac{\ln(x^2+1)}{\ln(x+1)}$
\end{enumerate}
\end{multicols}
\end{oefening}

\begin{oefening}
Bereken
\begin{enumerate}[(a)]
  \itemsep.7em
  \item $\displaystyle \left[\log_3 x^3\right]'$
  \item $\displaystyle \left[\log_3 3^x\right]'$
  \item $\displaystyle \left[\log \left(x^2+2x+1\right)\right]'$
  \item $\displaystyle \left[\ln \sqrt[3]{7x^3+3}\right]'$
\end{enumerate}
\end{oefening}

\begin{oefening}
{\em \scriptsize bron: Rekenregels afgeleide, auteur: Kathleen Hoornaert}\\
Zij
$$f(x)=\ln x \log x - \ln g \log_g x$$
met $g\in \mathbb{R}^+$. Dan is $f'(x)$ gelijk aan
\begin{enumerate}[(a)]
  \itemsep.8em
  \item $\frac{1}{x} \log x + \dfrac{\ln x}{x \ln 10} - \frac{1}{g}\log_g x - \frac{1}{x}$
  \item $\dfrac{\ln \frac{x^2}{10}}{x \ln 10}$
  \item $\dfrac{\log x + \ln x - 1}{x}$
\end{enumerate}
\end{oefening}

\begin{oefening}
Een Cessna vliegtuig stijgt op van een vliegveld op zeeniveau en de hoogte $h$ (in meter) at tijdstip $t$ (in minuten) wordt gegeven door
$$h=2000 \ln(t+1)$$
Bepaal de stijgingssnelheid op tijdstip $t=3\min$.
\end{oefening}
\vspace*{-1cm}

\pagebreak
\section{Afgeleide functie van een exponentiële functie}

\begin{oefening}
Bereken de afgeleide functie $Df(x)$
\begin{multicols}{2}
\begin{enumerate}[(a)]
  \itemsep0.8em
  \item $f(x)=\dfrac{e^x-1}{e^x+1}$
  \item $f(x)=(1+e^x)^3$
  \item $f(x)=\log(10+10^x)$
  \item $f(x)=\dfrac{1}{27}e^{3x}(9x^2-6x+2)$
  \item $f(x)=\dfrac{e^x-e^{-x}}{e^x+e^{-x}}$
  \item $f(x)=\ln\dfrac{e^x-1}{e^x+1}$
  \item $f(x)=3^{1+e^x}$
  \item $f(x)=e^{(3x^2-4)}$
  \item $f(x)=10^{\left(\dfrac{e^x-e^{-x}}{e^x+e^{-x}}\right)}$
  \item $f(x)=5^{\left(6x^2+\sin(3x^2)\right)}$
\end{enumerate}
\end{multicols}
\end{oefening}

\begin{oefening} 
De lading van een condensator in een circuit met een condensator condensator $C$, een weerstand $R$ en een spanningsbron $V$ wordt gegeven door
$$q=CV(1-e^{-t/RC})$$
Toon dat dit een oplossing is van de differentiaalvergelijking
$$Rq'+q/C=V$$
\end{oefening}

\begin{oefening}
Een computer wordt geprogrammeerd om rechthoeken weg te frezen van het eerste kwadrant onder de kromme
    $$y = e^{-x}$$
Wat is de oppervlakte van de grootste rechthoek die kan worden uitgefreesd?\\
\begin{center}
\definecolor{cqcqcq}{rgb}{0.75,0.75,0.75}
\begin{tikzpicture}[scale=3, line cap=round,line join=round,>=triangle 45,x=1.0cm,y=1.0cm]
\draw [color=cqcqcq,dash pattern=on 1pt off 1pt, xstep=0.5cm,ystep=0.5cm] (-0.57,-0.39) grid (4.16,1.32);
\draw[->,color=black] (-0.57,0) -- (4.16,0);
\foreach \x in {-0.5,0.5,1,1.5,2,2.5,3,3.5,4}
\draw[shift={(\x,0)},color=black] (0pt,2pt) -- (0pt,-2pt) node[below] {\footnotesize $\x$};
\draw[->,color=black] (0,-0.39) -- (0,1.32);
\foreach \y in {,0.5,1}
\draw[shift={(0,\y)},color=black] (2pt,0pt) -- (-2pt,0pt) node[left] {\footnotesize $\y$};
\draw[color=black] (0pt,-10pt) node[right] {\footnotesize $0$};
\clip(-0.57,-0.39) rectangle (4.16,1.32);
\fill[line width=1.2pt,fill=black,fill opacity=0.13] (0.1,0.1) -- (0.8,0.1) -- (0.8,0.4) -- (0.1,0.4) -- cycle;
\draw[line width=1.6pt, smooth,samples=100,domain=-0.5668076151119694:4.161079687635897] plot(\x,{2.718281828^(-(\x))});
\draw [line width=1.2pt] (0.1,0.1)-- (0.8,0.1);
\draw [line width=1.2pt] (0.8,0.1)-- (0.8,0.4);
\draw [line width=1.2pt] (0.8,0.4)-- (0.1,0.4);
\draw [line width=1.2pt] (0.1,0.4)-- (0.1,0.1);
\end{tikzpicture}
\end{center}
\end{oefening}


\pagebreak
\section{Afgeleide functie van een goniometrische functie}

\begin{oefening}
In éénzelfde assenstelsel, telkens in een ander kleur:
\begin{enumerate}[(a)]
  \item Teken de grafiek van de sinusfunctie. 
  \item Begin in de oorsprong en teken elke $(k\frac{\pi}{2}, f(k\frac{\pi}{2}))$ met $k\in\mathbb{Z}$ de raaklijn aan de sinusfunctie.
  \item Teken telkens een punt met als $x$-waarde de $x$-waarde van de raaklijn en als $y$-waarde de helling (rico) van deze raaklijn, teken dus $(k\frac{\pi}{2}, f'(k\frac{\pi}{2}))$.
  \item Verbind de net getekende punten met een vloeiende lijn.
\end{enumerate}
\end{oefening}

\begin{oefening}
De afgeleide functie van $f(x)=\cos(x)$ is $f'(x)=-\sin(x)$. Wat is de afgeleide functie van $g(x)=\cos(h(x))$ met $h(x)$ een reële functie?
\end{oefening}

\begin{oefening}
Leid af:
\begin{multicols}{2}
\begin{enumerate}[(a)]
  \itemsep0.5em
  \item $f(x)=\sin(x)+x$
  \item $f(x)=\sin(x)+2\cos(x)$
  \item $f(x)=\cos(2x)+4$
  \item $f(x)=-\dfrac{1}{3}\cos(3x)$
  \item $f(x)=x\cdot\sin(2x)$
  \item $f(x)=x^3\cdot\sin(x^2)$
  \item $f(x)=\cos(\dfrac{1}{x})$
  \item $f(x)=\sin(\sqrt{x})$
  \item $f(x)=\sin(\sqrt[4]{x})$
  \item $f(x)=\sqrt{\cos x}$
  \item $f(x)=\sqrt{\cos x+\sin x}$
  \item $f(x)=\dfrac{\sin x}{\cos x}$
  \item $f(x)=\dfrac{\sin x + \cos x^2}{\cos x}$
  \item $f(x)=\dfrac{\sin x + \cos^2 x}{\cos x}$
  \item $f(x)=\dfrac{\sin x + \cos(2x)}{\cos x}$
\end{enumerate}
\end{multicols}
\end{oefening}

\begin{oefening}
{\em \scriptsize Ijkingsproef industrieel ingenieur, ann de bodt, tanja van hecke}\\
Voor welke scherpe hoek $\theta$ is de afgelegde afstand
$$x=\dfrac{v^2_0 \sin 2\theta}{g}$$
langs de $x$-as maximaal, met $v_0$ de beginsnelheid en $g$ de valversnelling.
\begin{enumerate}[(A)]
  \itemsep.3em
  \item $0$
  \item $\dfrac{\pi}{4}$
  \item $\dfrac{\pi}{2}$
  \item $\dfrac{\pi}{3}$
\end{enumerate}
\end{oefening}

%\newpage
\end{document}







