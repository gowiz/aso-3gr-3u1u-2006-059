\documentclass[a4paper,12pt]{article}

\input{../gowiz.tex}

\usepackage{versions}
\excludeversion{theorie}


\begin{document}

\pagestyle{fancy}
\lhead{}
\rhead{Oefeningen Afgeleiden van bijzondere functies}

\begin{theorie}

\thispagestyle{empty}
\begin{center}
  \begin{mdframed}
  \centering
  \fontsize{40}{50}\selectfont Afgeleiden van bijzondere functies
  \end{mdframed}
  \vfill
  \vfill
\end{center}
\subsection*{Doelstelling}
Je kan de afgeleiden functie bepalen van\hfill  {\scriptsize(LP 2006-059, LI 1.6.13, 1.7.8, 1.8.6, 1.9.6)}
\begin{itemize}
  \item goniometrische functies
  \item exponentiële functies
  \item logaritmische functies
  \item irrationale functies
\end{itemize}


\pagestyle{empty}
\mbox{}
\newpage
\clearpage
\thispagestyle{empty}
%\mbox{}
\tableofcontents
\newpage
\clearpage
\pagenumbering{arabic}

\pagestyle{fancy}
\lhead{}
\rhead{Afgeleiden van bijzondere functies}

\end{theorie}

\onehalfspacing

\section{Afgeleide functie van een irrationale functie}

\begin{oefening}
Bereken
\begin{multicols}{2}
\begin{enumerate}[(a)]
  \itemsep0.5em
  \item $\displaystyle\left(2\cdot\sqrt{x}-1\right)'$
  \item $\displaystyle\left(\sqrt{3x+2}\right)'$
  \item $\displaystyle\left(\sqrt{2-4x}\right)'$
  \item $\displaystyle\left(\sqrt{x^2}\right)'$
  \item $\displaystyle\left(\sqrt{5x^2+2x-1}\right)'$
  \item $\displaystyle\left(\sqrt[3]{x}\right)'$
  \item $\displaystyle\left(\sqrt{x}+\dfrac{1}{\sqrt{x}}\right)'$
  \item $\displaystyle\left(\dfrac{1}{\sqrt{x^2+3}}\right)'$
  \item $\displaystyle\left(\sqrt[3]{\left(x+6\right)^2}\right)'$
  \item $\displaystyle\left(\sqrt[3]{x^2+6}\cdot\sqrt{x^2+6}\right)'$
  \item $\displaystyle\left(\dfrac{\sqrt[3]{x^2+2x+1}}{\sqrt[4]{x^2+2x+1}}\right)'$
  \item $\displaystyle\left(\sqrt[3]{2x-3}\cdot\sqrt{8x-12}\right)'$
  \item $\displaystyle\left(\dfrac{\sqrt{x+6}}{\sqrt[3]{x+2}}\right)'$
\end{enumerate}
\end{multicols}
\end{oefening}

\begin{oefening}
Bepaal de afgeleide functie $f'$ van\\
\begin{enumerate}[(a)]
  \itemsep0.5em
  \item $\displaystyle f(x)=4\cdot\sqrt{\frac{1}{2}x}$
  \item $\displaystyle f(x)=\sqrt{\frac{1}{2}x^3+\frac{1}{3}x^2+\frac{1}{4}x+\frac{1}{5}}$
  \item $\displaystyle f(x)=\sqrt{\dfrac{x^3+5x}{x^2-1}}$
  \item $\displaystyle f(x)=\sqrt[3]{\left(x^3+x^2+x+1\right)^4}$
  \item $\displaystyle f(x)=\sqrt[5]{\left(x^3+1\right)\left(x^2+1\right)}$
\end{enumerate}
\end{oefening}

\begin{oefening}
{\em \scriptsize Ijkingsproef industrieel ingenieur, ann de bodt, tanja van hecke}\\
Bereken de afgeleide van $u$ naar $x$ indien $u=r^{3/2}$ en $r=\sqrt{4+x^2}$.
\begin{enumerate}[(A)]
  \itemsep.3em
  \item $\dfrac{3}{2}\dfrac{x}{\sqrt{r}}$
  \item $\dfrac{3}{2}\sqrt{r}$
  \item $\dfrac{2}{5}r^\frac{5}{2}$
  \item $\dfrac{2}{5}xr^\frac{3}{2}$
\end{enumerate}
\end{oefening}

\pagebreak
\section{Afgeleide functie van een logaritmische functie}

\begin{oefening}
Bepaal de afgeleide functie $f'$ van\\[-1em]
\begin{enumerate}[(a)]
  \itemsep0.5em
  \item $\displaystyle f(x)=\log_5 x$
  \item $\displaystyle f(x)=\log_{3e} x$
  \item $\displaystyle f(x)=\ln(x-6)$
  \item $\displaystyle f(x)=\log(5x^2+4x+3)$
  \item $\displaystyle f(x)=\log^2 x$
  \item $\displaystyle f(x)=\dfrac{1}{\log x}$
  \item $\displaystyle f(x)=\log \sqrt[3]{\left(4x^2+2x+1\right)^2}$
  \item $\displaystyle f(x)=\sqrt[3]{\log\left(4x^2+2x+1\right)^2}$
  \item $\displaystyle f(x)=\sqrt[3]{\log^2(4x^2+2x+1)}$
  \item $\displaystyle f(x)=\log x^2 - \log x$
  \item $\displaystyle f(x)=\log 3x + \log x + \log x$
  \item $\displaystyle f(x)=\dfrac{\ln(x^2+1)}{\ln(x+1)}$
\end{enumerate}
\end{oefening}

\begin{oefening}
Bereken\\[-1em]
\begin{enumerate}[(a)]
  \itemsep0.5em
  \item $\displaystyle \left[\log_3 x^3\right]'$
  \item $\displaystyle \left[\log_3 3^x\right]'$
  \item $\displaystyle \left[\log \left(x^2+2x+1\right)\right]'$
  \item $\displaystyle \left[\ln \sqrt[3]{7x^3+3}\right]'$

\end{enumerate}
\end{oefening}

\pagebreak
\section{Afgeleide functie van een exponentiële functie}

\pagebreak
\section{Afgeleide functie van een goniometrische functie}


\begin{oefening}
{\em \scriptsize Ijkingsproef industrieel ingenieur, ann de bodt, tanja van hecke}\\
Voor welke scherpe hoek $\theta$ is de afgelegde afstand
$$x=\dfrac{v^2_0 \sin 2\theta}{g}$$
langs de $x$-as maximaal, met $v_0$ de beginsnelheid en $g$ de valversnelling.
\begin{enumerate}[(A)]
  \itemsep.3em
  \item $0$
  \item $\dfrac{\pi}{4}$
  \item $\dfrac{\pi}{2}$
  \item $\dfrac{\pi}{3}$
\end{enumerate}
\end{oefening}

%\newpage
\end{document}









