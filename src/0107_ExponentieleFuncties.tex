\documentclass[12pt,twoside]{article}

\input{../gowiz.tex}

\begin{document}

\begin{center}
  \begin{mdframed}
    \centering
    \fontsize{40}{60}\selectfont Exponentiële functies
  \end{mdframed}
  \vfill
  \includegraphics{exp_func} %TODO: replace
  \vfill
\end{center}

\subsection*{Doelstellingen}
\vspace*{-0.8cm}
\begin{singlespacing}
  Je \hfill  {\scriptsize(LP2006-059, LI1.9, ET32,14,31)}
  \begin{itemize}
    \itemsep-0.2em
  \item kan n-de wortel berekenen in $\mathbb{R}$
  \item kent de definitie van een macht met een rationale exponent en kunnen de elementaire rekenregels toepassen
  \item kan voor geschikte domeinen een verband leggen tussen de onderstaande functies en conclusies trekken in verband met hun grafieken:
    \begin{itemize}
    \item $x^2$ en $\sqrt{x}$
    \item $x^3$ en $\sqrt[3]{x}$
    \item $x^n$ en $\sqrt[n]{x}$
    \end{itemize}
  \item kent de definitie van een exponentiële functie $f(x)=a^x$
  \item kan het onderscheid tussen een lineair en een exponentieel groeiproces
  \item kunt van een exponentiële functie de tabel, de grafiek, het domein, enkele bijzondere waarden, het stijgen/dalen en het asymptotisch gedrag bepalen, eventueel met behulp van ICT
  \item kunt aan de hand van de grafiek het domein, het bereik, de bijzondere waarden, het tekenverloop, het stijgen/dalen en het asymptotisch gedrag bepalen van exponentiële functies
  \end{itemize}
\end{singlespacing}

\thispagestyle{empty}
\newpage

\tableofcontents

\vfill

De hoofdstukken over machten en machtswortels zijn met toestemming gebaseerd op {\em Paul's Online Notes} die men kan terugvinden op \url{https://tutorial.math.lamar.edu/}. Deze site bij je favorieten plaatsen kan nuttig zijn.

\thispagestyle{empty}
\newpage

\pagenumbering{arabic}

\pagestyle{fancy}
\fancyhead[RO,LE]{Exponentiële functies}
\fancyhead[RE,LO]{}

\section{Machten met gehele exponenten}

\subsection{Machten waarbij de exponenten strikt positief zijn}

We beginnen dit hoofdstuk met machten waarbij de exponenten niet alleen gehele getallen zijn, maar ook nog strikt positief. We bekijken exponenten die nul zijn of negatief zo meteen.

\paragraph*{Definitie}
\begin{mdframed}
  Als $a$ een reëel getal is en $n$ is een strikt positief geheel getal, dan geldt
  $$a^n = \underbrace{a \cdot a \cdot a \cdot \cdots \cdot a}_{n \text{ factoren}}$$
  Het getal $a$ noemen we het {\bf grondtal}, het getal $n$ noemen we de {\bf exponent}. De uitkomst $a^n$ is de {\bf macht}.
\end{mdframed}

\paragraph*{Voorbeeld:} $3^5 = 3 \cdot 3 \cdot 3 \cdot 3 \cdot 3 = 243$

\paragraph*{Merk op:} Herinner je nog dat de exponent inwerkt op hetgeen onmiddellijk links van de exponent staat. Dit is vooral van belang als het grondtal negatief is. Zo zullen we volgende twee gevallen beschouwen
$$(-2)^4 \qquad\qquad -2^4$$

In het eerste geval werkt de exponent in op alles tussen de haakjes. Het grondtal is met andere woorden $-2$. We krijgen in dit geval dus
$$(-2)^4=(-2)(-2)(-2)(-2)=16$$

In het tweede geval werkt de exponent enkel in op $2$. De macht is dus $2^4$. Het min teken blijft dus voor deze macht. De min heeft dus NIETS te maken met de macht! We berekenen dus
$$-2^4=- 2 \cdot 2 \cdot 2 \cdot 2 = -16$$

\subsection{Machten met nul als exponent}

\paragraph*{Definitie}
\begin{mdframed}
  Er geldt
  $$a^0 = 1 \text{ als } a \neq 0$$
\end{mdframed}

\paragraph*{Merk op:} Het grondtal $a$ mag hier geen nul zijn. Dit is belangrijk, want $0^0$ is niet gedefinieerd. Probeer gerust eens uit te rekenen met je rekentoestel. Met de rekenregels voor machten kan aangetoond worden dat $0^0$ inderdaad niet zinvol is.

\paragraph*{Voorbeeld:} $\left(-1234\right)^0 = 1$

\subsection{Machten met een strikt negatieve gehele exponent}

\paragraph*{Definitie}
\begin{mdframed}
  Als $a$ een reëel getal verschillend van nul is en $n$ is een strikt positief geheel getal (ja, positief!), dan geldt
  $$a^{-n} = \dfrac{1}{a^n}$$
\end{mdframed}

\paragraph*{Merk op:} In bovenstaande definitie mag $a$ niet nul zijn, zie je in waarom? Herinner jou dat delen door nul niet toegestaan is en als we hadden toegestaan dat $a$ nul was, dan zouden we hier nu delen door 0.

\paragraph*{Voorbeelden:}
$5^{-2}=\dfrac{1}{5^2} = \dfrac{1}{25} \qquad \qquad (-4)^{-3} = \dfrac{1}{(-4)^3} = \dfrac{1}{-64}=-\dfrac{1}{64}$

\subsection{Eigenschappen voor machten}

Hier zijn een aantal eigenschappen voor machten. We gaan er steeds van uit dat de waarden zinvol zijn. Zo is $a\in\mathbb{R}$ of $a\in\mathbb{R}_0$ als delen door 0 zou kunnen voorkomen.

\subsubsection*{Product van machten}
$$a^n a^m = a^{n+m} \qquad \text{Vb: } a^{-9}a^4 = a^{-5}=\dfrac{1}{a^5}$$

\subsubsection*{Quotiënt van machten}
$$\dfrac{a^n}{a^m} = a^{n-m} \qquad \text{Vb: } \dfrac{a^4}{a^{11}}=a^{4-11}=a^{-7}=\dfrac{1}{a^7}$$

\subsubsection*{Macht van een macht}
$$\left(a^n\right)^m = a^{nm}\qquad \text{Vb: } \left(a^7\right)^3=a^{7\cdot 3}=a^{21}$$

\subsubsection*{Macht van een product}
$$\left(ab\right)^n = a^nb^n \qquad \text{Vb: } \left(ab\right)^{-4}=a^{-4}b^{-4}=\dfrac{1}{a^4}\dfrac{1}{b^4}=\dfrac{1}{a^4b^4}$$

\subsubsection*{Macht van een quotiënt}
$$\left(\dfrac{a}{b}\right)^n = \dfrac{a^n}{b^n} \qquad \text{Vb: } \left(\dfrac{a}{b}\right)^8 = \dfrac{a^8}{b^8}$$

\subsubsection*{Macht van een quotiënt met een negatieve exponent}
$$\left(\dfrac{a}{b}\right)^{-n} = \left(\dfrac{b}{a}\right)^{n} \qquad \text{Vb: } \left(\dfrac{a}{b}\right)^{-10} = \left(\dfrac{b}{a}\right)^{10} = \dfrac{b^{10}}{a^{10}}$$

\paragraph{Merk op:} Veel van bovenstaande eigenschappen werden gegeven met maar twee termen of factoren. Ze kunnen gerust uitgebreid worden naar meerdere termen of factoren. Bijvoorbeeld $$(abcd)^n=a^nb^nc^nd^n$$

\paragraph{Veel voorkomende fout:} Er worden snel fouten gemaakt bij het rekenen met machten. Laten we eens een paar voorbeelden bespreken.

\begin{align*}
  \text{Juist: } ab^{-2} &= a\dfrac{1}{b^2} = \dfrac{a}{b^2}\\
  \text{Fout:  } ab^{-2} &\neq \color{gray}{\dfrac{1}{ab^2}}
\end{align*}

De exponent werkt enkel in op hetgeen onmiddellijk links van de exponent staat. In dit geval enkel op de $b$. Je mag dus de $a$ NIET meenemen naar de noemer. Vergelijk dit met het volgende voorbeeld:
$$ (ab)^{-2} = \dfrac{1}{(ab)^2} $$

Een gelijkaardig voorbeeld is
\begin{align*}
  \text{Juist: } \dfrac{1}{3a^{-5}} &= \dfrac{1}{3}\dfrac{1}{a^{-5}} = \dfrac{1}{3}a^5 = \dfrac{a^5}{3}\\
  \text{Fout:  } \dfrac{1}{3a^{-5}} &\neq \color{gray}{3a^5}
\end{align*}

Vergelijk dit met
$$\dfrac{1}{(3a)^{-5}} = (3a)^5 = 3^5 a^5 = 243 a^5$$

\paragraph{Afspraak:} Het is de gewoonte om in het eindantwoord altijd de machten te schrijven met een positieve exponent. Niets houd je tegen om in tussenstappen negatieve exponenten te gebruiken als dit de berekeningen eenvoudiger maakt.


\begin{oefening}
  Vereenvoudig elk van de volgende machten
  \begin{multicols}{2}
    \begin{enumerate}[(a)]
      \itemsep1em
    \item $\displaystyle {\left( {4{x^{ - 4}}{y^5}} \right)^3}  $
    \item $\displaystyle {\left( { - 10{z^2}{y^{ - 4}}} \right)^2}{\left( {{z^3}y} \right)^{ - 5}}  $
    \item $\displaystyle \displaystyle \frac{{{n^{ - 2}}m}}{{7{m^{ - 4}}{n^{ - 3}}}}  $
    \item $\displaystyle \displaystyle \frac{{5{x^{ - 1}}{y^{ - 4}}}}{{{{\left( {3{y^5}} \right)}^{ - 2}}{x^9}}}  $
    \item $\displaystyle {\left( {\displaystyle \frac{{{z^{ - 5}}}}{{{z^{ - 2}}{x^{ - 1}}}}} \right)^6}  $
    \item $\displaystyle {\left( {\displaystyle \frac{{24{a^3}{b^{ - 8}}}}{{6{a^{ - 5}}b}}} \right)^{ - 2}}  $
    \end{enumerate}
  \end{multicols}
\end{oefening}

\begin{oefening}
  Bereken zonder \zrm{ZRM}. Geef het antwoord als één enkel getal zonder exponenten. Rationale getallen worden in breukvorm gegeven. Reële getallen worden in wortelvorm gegeven.
  \begin{multicols}{2}
    \begin{enumerate}[(a)]
      \itemsep1em
    \item $ - {6^2} + 4 \cdot {3^2} $
    \item $ \displaystyle \frac{{{{\left( { - 2} \right)}^4}}}{{{{\left( {{3^2} + {2^2}} \right)}^2}}} $
    \item $ \displaystyle \frac{{{4^0} \cdot {2^{ - 2}}}}{{{3^{ - 1}} \cdot {4^{ - 2}}}} $
    \item $ {2^{ - 1}} + {4^{ - 1}} $
    \end{enumerate}
  \end{multicols}
\end{oefening}

\begin{oefening}
  Vereenvoudig volgende uitdrukkingen. Gebruik in je antwoord enkel nog positieve exponenten.
  \begin{multicols}{2}
    \begin{enumerate}[(a)]
      \itemsep1em
    \item $ {\left( {2{w^4}{v^{ - 5}}} \right)^{ - 2}}  $
    \item $ \displaystyle \frac{{2{x^4}{y^{ - 1}}}}{{{x^{ - 6}}{y^3}}}  $
    \item $ \displaystyle \frac{{{m^{ - 2}}{n^{ - 10}}}}{{{m^{ - 7}}{n^{ - 3}}}}  $
    \item $ \displaystyle \frac{{{{\left( {2{p^2}} \right)}^{ - 3}}{q^4}}}{{{{\left( {6q} \right)}^{ - 1}}{p^{ - 7}}}}  $
    \item $ {\left( {\displaystyle \frac{{{z^2}{y^{ - 1}}{x^{ - 3}}}}{{{x^{ - 8}}{z^6}{y^4}}}} \right)^{ - 4}}  $
    \end{enumerate}
  \end{multicols}
\end{oefening}

\subsection{Extra oefeningen}

\begin{oefening} Begrijp je de theorie?
  \begin{enumerate}[(a)]
  \item Geef de definitie van een macht met een natuurlijke exponent.
  \item Bespreek het teken van machten met een natuurlijke exponent
  \item Geef de definitie van een macht met een negatieve exponent.
  \end{enumerate}
\end{oefening}

\begin{oefening}*
Leid volgende eigenschappen van de machtsverheffing af:
\begin{align*}
  a^m a^n &= a^{m+n} && \mbox{product van machten}\\
  \dfrac{a^m}{a^n} &= a^{m-n} && \mbox{quotiënt van machten}\\
  \left(a^m\right)^n &= a^{m\cdot n} && \mbox{macht van een macht}\\
  \left(ab\right)^n &= a^n b^n && \mbox{macht van een product}\\
  \left(\dfrac{a}{b}\right)^n &= \dfrac{a^n}{b^n} && \mbox{macht van een quotiënt}
\end{align*}
\end{oefening}

\begin{oefening}
Bereken zonder \zrm{ZRM}.
\begin{multicols}{4}
  \begin{enumerate}[(a)]
    \itemsep1em
    \item $5^2$
    \item $5^3$
    \item $(-5)^2$
    \item $(-5)^3$
    \item $-5^2$
    \item $-5^3$
    \item $5^{-2}$
    \item $5^{-3}$
    \item $-5^{-2}$
    \item $-5^{-3}$
    \item $(-5)^{-2}$
    \item $(-5)^{-3}$
  \end{enumerate}
\end{multicols}
\end{oefening}

\begin{oefening}
  Bereken zonder \zrm{ZRM}. Geef het antwoord als één enkel getal zonder exponenten. Rationale getallen worden in breukvorm gegeven. Reële getallen worden in wortelvorm gegeven.
  \begin{multicols}{3}
    \begin{enumerate}[(a)]
      \itemsep1em
    \item $2 \cdot {5^2} + {\left( { - 4} \right)^2}$
    \item ${6^0} - {3^5}$
    \item $3 \cdot {4^3} + 2 \cdot {3^2}$
    \item ${\left( { - 1} \right)^4} + 2{\left( { - 3} \right)^4}$
    \item ${7^0}{\left( {{4^2} \cdot {3^2}} \right)^2}$
    \item $- {4^3} + {\left( { - 4} \right)^3}$
    \item $8 \cdot {2^{ - 3}} + {16^0}$
    \item ${\left( {{2^{ - 1}} + {3^{ - 1}}} \right)^{ - 1}}$
    \item $\displaystyle \frac{{{3^2} \cdot {{\left( { - 2} \right)}^3}}}{{{6^{ - 2}}}}$
    \item $\displaystyle \frac{{{4^{ - 2}} \cdot {5^3}}}{{{3^{ - 4}}}}$
    \end{enumerate}
  \end{multicols}
\end{oefening}

\begin{oefening}
Vereenvoudig volgende uitdrukkingen. Gebruik in je antwoord enkel nog positieve exponenten.
\begin{multicols}{2}
  \begin{enumerate}[(a)]
    \itemsep1em
  \item ${\left( {3{x^{ - 2}}{y^{ - 4}}} \right)^{ - 1}}$
  \item ${\left( {{{\left( {2{a^2}} \right)}^{ - 3}}{b^4}} \right)^{ - 3}}$
  \item $\displaystyle \frac{{{c^{ - 6}}{b^{10}}}}{{{b^9}{c^{ - 11}}}}$
  \item $\displaystyle \frac{{4{a^3}{{\left( {{b^2}a} \right)}^{ - 4}}}}{{{c^{ - 6} 7}}}$
\item $\displaystyle \frac{{{{\left( {6{v^2}} \right)}^{ - 1}}{w^{ - 4}}}}{{{{\left( {2v} \right)}^{ - 3}}{w^{10}}}}$
\item ${\left( {\displaystyle \frac{{{{\left( {8{x^{21}}} \right)}^0}{y^{ - 3}}{x^8}}}{{{y^{ - 9}}{x^{ - 1}}}}} \right)^6}$
\item ${\left( {\displaystyle \frac{{{a^2}{b^{ - 4}}{c^{ - 1}}}}{{{b^{ - 9}}{c^8}{a^{ - 4}}}}} \right)^{ - 2}}$
\item ${\left( {\displaystyle \frac{{{p^{ - 6}}{q^7}{{\left( {{p^2}q} \right)}^{ - 3}}}}{{{{\left( {{p^{ - 1}}{q^{ - 4}}} \right)}^2}{p^{10}}}}} \right)^3}$
\end{enumerate}
\end{multicols}
\end{oefening}

\begin{oefening}
  Zijn volgende beweringen waar of vals?
  \begin{enumerate}[(a)]
    \itemsep1em
  \item $\displaystyle \frac{1}{{6x}} = 6{x^{ - 1}}$ \hfill \texttt{WAAR / VALS}\mbox{\hspace{5cm}}
  \item ${\left( {{x^3}} \right)^7} = {x^{10}}$ \hfill \texttt{WAAR / VALS}\mbox{\hspace{5cm}}
  \item ${\left( {{m^3}{n^4}} \right)^2} = {m^{12}}{n^8}$ \hfill \texttt{WAAR / VALS}\mbox{\hspace{5cm}}
  \item ${\left( {{{\left( {{z^2}} \right)}^3}} \right)^4} = {z^{24}}$ \hfill \texttt{WAAR / VALS}\mbox{\hspace{5cm}}
  \item ${\left( {x + y} \right)^3} = {x^3} + {y^3}$ \hfill \texttt{WAAR / VALS}\mbox{\hspace{5cm}}
  \end{enumerate}
\end{oefening}

\pagebreak
\section{Machten met rationale exponenten}

We weten nu hoe we met machten met gehele exponenten moeten werken. Laten we eens een stapje verder gaan en naar machten met rationale exponenten kijken. Dit zijn exponenten van de vorm
$$b^{\frac{m}{n}}$$
met $m$ en $n$ gehele getallen.

\subsection{Vereenvoudigd geval}

We beginnen eenvoudig met
$$b^{\frac{1}{n}}$$
met $n$ een geheel getal. In dit geval kunnen we exponenten van deze vorm definiëren als
\paragraph{Definitie}
\begin{mdframed}
  $$a=b^{\frac{1}{n}} \quad\Leftrightarrow\quad a^n=b$$
\end{mdframed}
Als we dus $b^{1/n}$ berekenen, dan vragen we ons af welk nummer (in dit geval $a$) we tot de macht $n$ moeten verheffen om $b$ te krijgen.

\paragraph{Benaming:} Vaak noemen we $b^{1/n}$ de $n$-de machtswortel van $b$. Meer hierover verder.

\begin{oefening}
  Bereken elk van volgende machten.
  \begin{multicols}{3}
    \begin{enumerate}[(a)]
      \itemsep1em
    \item $25^{\frac{1}{2}}$
    \item $32^{\frac{1}{5}}$
    \item $81^{\frac{1}{4}}$
    \item $(-8)^{\frac{1}{3}}$
    \item $(-16)^{\frac{1}{4}}$
    \item $-16^{\frac{1}{4}}$
    \end{enumerate}
  \end{multicols}
\end{oefening}

Achteraan in de bijlage zit een tabel met machten die je kunnen helpen dit soort oefeningen op te lossen. Maak je niet te veel zorgen als het niet lukt om deze machten op te lossen uit je hoofd, de tabel kan je gebruiken op toetsen en examens.

Alle eigenschappen voor machten met gehele exponent blijven ook geldig voor machten met rationale exponent. Er moeten dus geen nieuwe rekenregels geleerd worden.

\subsection{Algemeen geval}

In het algemeen geval kunnen we de macht herschrijven tot het vereenvoudigd geval m.b.v. macht van een macht:
$$b^{\frac{m}{n}} = \left(b^{\frac{1}{n}}\right)^m \text{ of } \left(b^m\right)^{\frac{1}{n}}$$

Vaak is het zelfs niet nodig om de macht te herschrijven. Maken we gebruik van de eigenschappen van machten, en herschrijven we 'grote' getallen als een macht, dan worden de berekeningen vaak heel erg eenvoudig:
\begin{align*}
  8^{\frac{2}{3}} &= \left(2^3\right)^{\frac{2}{3}}\\
                  &= 2^{3\cdot \frac{2}{3}}\\
                  &= 2^2\\
                  &= 4
\end{align*}

\begin{oefening}
  Bereken of vereenvoudig elk van volgende machten.
  \begin{multicols}{2}
    \begin{enumerate}[(a)]
      \itemsep1em
    \item $ {625^{\frac{3}{4}}}  $
    \item $ {\left( {\displaystyle \frac{{243}}{{32}}} \right)^{\frac{4}{5}}}  $
    \item $ {\left( {\displaystyle \frac{{{w^{ - 2}}}}{{16{v^{\frac{1}{2}}}}}} \right)^{\frac{1}{4}}}  $
    \item $ {\left( {\displaystyle \frac{{{x^2}{y^{ - \frac{2}{3}}}}}{{{x^{ - \frac{1}{2}}}{y^{ - 3}}}}} \right)^{ - \frac{1}{7}}}  $
    \end{enumerate}
  \end{multicols}
\end{oefening}

\paragraph{Veel voorkomende fout:} Nog een waarschuwing voor een veel voorkomende fout, verwar negatieve exponenten niet met rationale exponenten:

\begin{align*}
  \text{Juist: }b^{-n} &= \dfrac{1}{b^n}\\
  \text{Fout:  } b^{-n} &\neq \color{gray}{b^{\frac{1}{n}}}
\end{align*}

\begin{oefening}
  Bereken zonder \zrm{ZRM} en geef je antwoord als één enkel getal zonder exponenten.
  \begin{multicols}{3}
    \begin{enumerate}[(a)]
      \itemsep1em
    \item $ {36^{\frac{1}{2}}}  $
    \item $ {\left( { - 125} \right)^{\frac{1}{3}}}  $
    \item $ - {16^{\frac{3}{2}}}  $
    \item $ {27^{ -\frac{5}{3}}}  $
    \item $ {\displaystyle \left( {\frac{9}{4}} \right)^{\frac{1}{2}}}  $
    \item $ {\displaystyle \left( {\frac{8}{{343}}} \right)^{ - \frac{2}{3}}}  $
    \end{enumerate}
  \end{multicols}
\end{oefening}


\begin{oefening}
  Vereenvoudig en laat in het antwoord geen negatieve exponenten staan.
  \begin{multicols}{2}
    \begin{enumerate}[(a)]
      \itemsep1em
    \item $ {\left( {{a^3}\,{b^{ - \frac{1}{4}}}} \right)^{\frac{2}{3}}}  $
    \item $ {\displaystyle \left( {\frac{{{q^3}\,{p^{ - \frac{1}{2}}}}}{{{q^{ - \frac{1}{3}}}\,p}}} \right)^{\frac{3}{7}}}  $
    \item $ {x^{\frac{1}{4}}}\,{x^{ - \frac{1}{5}}}  $
    \item $ {\displaystyle \left( {\frac{{{m^{\frac{1}{2}}}\,{n^{ - \frac{1}{3}}}}}{{{n^{\frac{2}{3}}}\,{m^{ - \frac{7}{4}}}}}} \right)^{ - \frac{1}{6}}}  $
    \end{enumerate}
  \end{multicols}
\end{oefening}

\subsection{Extra oefeningen}

\begin{oefening}
  Bereken zonder \zrm{ZRM}. Geef het antwoord als één enkel getal zonder exponenten.
  \begin{multicols}{3}
    \begin{enumerate}[(a)]
      \itemsep1em
    \item ${64^{\frac{1}{2}}}$
    \item $- {64^{\frac{1}{2}}}$
    \item ${16^{\frac{1}{2}}}$
    \item ${16^{\frac{1}{4}}}$
    \item ${\left( { - 243} \right)^{\frac{1}{5}}}$
    \item ${121^{ - \,\,\frac{1}{2}}}$
    \item ${\left( { - 64} \right)^{ - \,\,\frac{1}{3}}}$
    \item ${\left( { \displaystyle \frac{{625}}{{256}}} \right)^{\frac{1}{4}}}$
    \item ${\left( { \displaystyle  - \frac{{27}}{8}} \right)^{\frac{1}{3}}}$
    \item ${49^{\frac{5}{2}}}$
    \item ${64^{ - \,\,\frac{5}{6}}}$
    \item ${\left( { - 729} \right)^{\frac{4}{3}}}$
    \item ${\left( { \displaystyle \frac{{121}}{{36}}} \right)^{ - \,\,\frac{3}{2}}}$
    \item ${\left( { \displaystyle  - \frac{{32}}{{243}}} \right)^{\frac{2}{5}}}$
    \item ${\left( { \displaystyle \frac{{81}}{{625}}} \right)^{\frac{3}{4}}}$
    \end{enumerate}
  \end{multicols}
\end{oefening}


\begin{oefening}
  Vereenvoudig volgende uitdrukkingen. Gebruik in je antwoord enkel nog positieve exponenten.
  \begin{multicols}{3}
  \begin{enumerate}[(a)]
    \itemsep1em
  \item ${\left( {{p^{ - 2}}{q^{ - 4}}} \right)^{\frac{3}{2}}}$
  \item ${x^{\frac{3}{4}}}{\left( {{x^2}\,{x^{ - \,\,\frac{1}{3}}}} \right)^{\frac{3}{2}}}$
  \item ${a^{\frac{1}{2}}}\,{a^{ - \,\,\frac{1}{3}}}\,{a^{\frac{1}{4}}}$
  \item ${\left( {{m^{ - \,\,\frac{7}{3}}}{n^{\frac{5}{4}}}} \right)^{ - \,\,\frac{8}{9}}}$
  \item ${\left( { \displaystyle \frac{{{a^{ - \,\,\frac{1}{3}}}\,{b^2}}}{{{b^{\frac{2}{3}}}\,{a^{ - \,\,\frac{3}{4}}}}}} \right)^{\frac{1}{5}}}$
  \item ${\left( { \displaystyle \frac{{{p^{\frac{1}{2}}}\,{q^{\frac{1}{3}}}}}{{{p^{ - \,\,\frac{1}{3}}}{q^{ - \,\,\frac{1}{4}}}}}} \right)^{ - 3}}$
  \item ${\left( { \displaystyle \frac{{{x^{\frac{3}{4}}}\,{y^{ - \,\,\frac{2}{3}}}}}{{{x^{\frac{7}{4}}}}}} \right)^{\frac{7}{8}}}$
  \item ${\left( { \displaystyle \frac{{{b^3}\,{c^{ - \,\,\frac{1}{4}}}\,{a^{ - 1}}}}{{{b^{\frac{1}{4}}}\,{a^{ - \,\,\frac{2}{7}}}{c^{\frac{3}{2}}}}}} \right)^{\frac{2}{3}}}$
  \end{enumerate}
  \end{multicols}
\end{oefening}


\begin{oefening}
  Zijn volgende beweringen waar of vals?
  \begin{enumerate}[(a)]
    \itemsep1em
  \item ${a^{ - \,\,\frac{3}{2}}} = {a^{\frac{2}{3}}}$ \hfill \texttt{WAAR / VALS}\mbox{\hspace{5cm}}
  \item ${x^{ - n}} = {x^{\frac{1}{n}}}$ \hfill \texttt{WAAR / VALS}\mbox{\hspace{5cm}}
  \end{enumerate}
\end{oefening}

\pagebreak
\section{Machtswortels}

\subsection{Definitie}

\begin{mdframed}
  Als $n$ een natuurlijk getal groter dan 1 is en $a$ in een reëel getal, dan definiëren we
  \[\sqrt[n]{a} = {a^{\frac{1}{n}}}\]
  We noemen $n$ de \textbf{(wortel)index} en $a$ het \textbf{(wortel)grondtal}, het symbool $\sqrt{\mbox{ }}$ wordt de \textbf{machtswortel} genoemd.
\end{mdframed}

Uit deze definitie blijkt dat de machtswortel een andere manier is om machten met rationale exponent op te schrijven.

\paragraph*{Merk op:} De index moet steeds bij het machtswortelsymbool staan. We zien dit als één enkel symbool.

\paragraph*{Uitzondering:} De \textbf{vierkantswortel} $\sqrt{\mbox{ }}$ is de machtswortel met wortelindex $2$, m.a.w.:
\[\sqrt[2]{a} = \sqrt a \]

\begin{oefening}
  Schrijf elk van volgende machtswortels als machten met rationale exponent
  \begin{enumerate}[(a)]
    \itemsep1em
  \item \(\sqrt[4]{{16}}\)
  \item \(\sqrt[{10}]{{8x}}\)
  \item \(\sqrt {{x^2} + {y^2}} \)
\end{enumerate}
\end{oefening}

\paragraph*{Veel voorkomende fout:} We moeten terug opletten met haakjes. Wanneer we een uitdrukking van machtswortel naar macht omzetten die meerdere termen of factoren bevat, dan moeten het juiste wortelargument tussen haakjes plaatsen. Beschouw
\[8{x^{\frac{1}{{10}}}}\]

Uit het voorafgaande weten we dat de exponent enkel werkt op de factor die onmiddellijk gevolgd wordt door de exponent. Dus omzetten geeft
\[8{x^{\frac{1}{{10}}}} = 8\cdot\sqrt[{10}]{x} \ne \sqrt[{10}]{{8x}}\]

\subsection{Vereenvoudigen van machtswortels}

Zet machtswortels altijd eerst om in machten, vereenvoudig dan de machten met de eigenschappen van machten. Als er na het vereenvoudigen nog machten met rationale exponenten over blijven, dan kan je deze nog omzetten in machtswortels. In je tussenstappen vermijd je dus machtswortels, in je eind antwoord kunnen er wel machtswortels staan.

\paragraph*{Voorbeelden:} We beginnen met een eenvoudig voorbeeld om het principe te tonen. Het lijkt verleidelijk om de tussenstappen over te slaan en met machtswortels te blijven werken. De kracht om over te gaan naar machten wordt gaandeweg als de opgave moeilijker is, wel duidelijk.
\begin{enumerate}[(a)]
\item \mbox{}\vspace{-2em}
  \begin{align*}
    \sqrt {16} &= {16^{\frac{1}{2}}}\\
               &= \left(4^2\right)^{\frac{1}{2}}\\
               &= 4
  \end{align*}
\item \mbox{}\vspace{-2em}
  \begin{align*}
    \sqrt[4]{16} &= {16^{\frac{1}{4}}}\\
               &= \left(2^4\right)^{\frac{1}{4}}\\
               &= 2
  \end{align*}
\item \mbox{}\vspace{-2em}
  \begin{align*}
    \sqrt[3] {16} &= {16^{\frac{1}{3}}}\\
               &= \left(2^4\right)^{\frac{1}{3}}\\
               &= \left(2^3\cdot 2\right)^{\frac{1}{3}}\\
               &= 2 \cdot 2^{\frac{1}{3}}\\
               &= 2 \cdot \sqrt[3]{2}
  \end{align*}
\item \mbox{}\vspace{-2em}
  \begin{align*}
    \sqrt[5] {16^3\cdot\sqrt[3]{16}} &= \left(16^3\cdot 16^{1/3}\right)^{1/5}\\
               &= 16^{3/5}\cdot 16^{1/15}\\
               &= 16^{3/5+1/15}\\
               &= 16^{10/15}\\
               &= 16^{1/3}\\
               &= 2 \cdot \sqrt[3]{2}
  \end{align*}
\end{enumerate}

\paragraph*{Opmerking: }
Hierboven beweren we dat $\sqrt{16} = 4$ dit omdat we $16=4^2$. Maar $4$ is niet het enige getal dat we kunnen kwadrateren om $16$ uit te komen! Er geldt ook $16=(-4)^2$. Dus waarom hebben we dan niet als antwoord $-4$? Of beide? Goede vraag! Het is de algemene afspraak, dat als we \textbf{machten met even index vereenvoudigen, dat we dan altijd het positieve antwoord} nemen. Als we het negatieve antwoord willen, dan schrijven we $$- \sqrt {16} = - 4$$.

Merk ook op dat dit probleem zich niet stelt met machtswortels waarbij de wortelindex oneven is. Daar hebben we altijd maar één enkel antwoord.

\begin{oefening}
  Vereenvoudig
  \begin{multicols}{2}
    \begin{enumerate}[(a)]
      \itemsep1em
    \item \(\sqrt[5]{{243}}\)
    \item \(\sqrt[4]{{1296}}\)
    \item \(\sqrt[3]{{ - 125}}\)
    \item \(\sqrt[4]{{ - 16}}\)
    \end{enumerate}
  \end{multicols}
\end{oefening}

\paragraph*{Opmerking: }
We kunnen machtswortels met een negatief grondtal niet uitrekenen als de wortelindex even is, wel als deze oneven is.

\subsection{Vereenvoudigde wortelvorm}

\paragraph{Definitie}
\begin{mdframed}
  We noemen de \textbf{vereenvoudigde wortelvorm} de machtswortel waarbij elk van volgende voorwaarde vervuld is:
  \begin{itemize}
  \item Alle exponenten in het wortelargument moeten kleiner zijn dan de wortelindex.
  \item Geen enkele exponent in het wortelargument mag nog een factor gemeenschappelijk hebben met de wortelindex.
  \item Er bevinden zich geen breuken in het wortelargument.
  \item Er bevinden zich geen machtswortels in de noemer van een breuk.
  \end{itemize}
\end{mdframed}

\subsection{Eigenschappen van machtswortels}

Als $n\in\mathbb{N}\setminus{0,1}$ en $a,b\in\mathbb{R}^+$
\begin{align*}
  \sqrt[n]{{{a^n}}} &= a\\
  \sqrt[n]{{ab}} &= \sqrt[n]{a}\,\sqrt[n]{b}\\
  \displaystyle \sqrt[n]{{\frac{a}{b}}} &= \frac{{\sqrt[n]{a}}}{{\sqrt[n]{b}}}
\end{align*}

\paragraph*{Merk op: } Doordat we machtswortels kunnen schrijven als machten, hebben bovenstaande eigenschappen weinig belang. Je kan alle berekeningen doen zonder deze definities.

\paragraph*{Veel voorkomende fout:}
We kunnen producten en quotiënten onder een machtswortel splitsen. Maar we kunnen hetzelfde niet doen bij een som of een verschil!
\[\sqrt[n]{{a + b}} \ne {\color{gray}\sqrt[n]{a} + \sqrt[n]{b}}\qquad \text{EN}\qquad\sqrt[n]{{a - b}} \ne {\color{gray}\sqrt[n]{a} - \sqrt[n]{b}}\]

Controleer maar eens met een eenvoudig getalvoorbeeld:
\[5 = \sqrt {25} = \sqrt {9 + 16} \ne \sqrt 9 + \sqrt {16} = 3 + 4 = 7\]



\subsection{Som en verschil met machtswortels}

Het optellen en aftrekken met machtswortels verloopt analoog als dat met veeltermen. Herinner je dat we bij het optellen en aftrekken van alle gelijksoortige termen met een $x$, we alle coëfficiënten moeten optellen en aftrekken horende bij de gelijksoortige termen, bijvoorbeeld

\[4x + 9x = \left( {4 + 9} \right)x = 13x\hspace{0.25in}\hspace{0.25in}3x - 11x = \left( {3 - 11} \right)x = - 8x\]

Het optellen en aftrekken van machtswortels werkt net op dezelfde manier, bijvoorbeeld

\[4\sqrt x + 9\sqrt x = \left( {4 + 9} \right)\sqrt x = 13\sqrt x\]
\[3\,\,\sqrt[{10}]{5} - 11\,\,\sqrt[{10}]{5} = \left( {3 - 11} \right)\sqrt[{10}]{5} = - 8\,\,\sqrt[{10}]{5}\]

\begin{oefening}
  Vereenvoudig volgende producten, veronderstel dat $x$ positief is
  \begin{enumerate}[(a)]
    \itemsep1em
  \item \(\left( {\sqrt x + 2} \right)\left( {\sqrt x - 5} \right)\)
  \item \(\left( {3\,\sqrt x - \sqrt y } \right)\left( {2\sqrt x - 5\sqrt y } \right)\)
  \item \(\left( {5\sqrt x + 2} \right)\left( {5\sqrt x - 2} \right)\)
  \end{enumerate}
\end{oefening}

\begin{oefening}
  Vereenvoudig (schrijf dus alle machtswortels in hun vereenvoudigde wortelvorm):
  \begin{multicols}{3}
    \begin{enumerate}[(a)]
      \itemsep1em
    \item \(\sqrt {{y^7}} \)
    \item \(\sqrt[9]{{{x^6}}}\)
    \item \(\sqrt {18{x^6}{y^{11}}} \)
    \item \(\sqrt[4]{{32{x^9}{y^5}{z^{12}}}}\)
    \item \(\sqrt[5]{{{x^{12}}{y^4}{z^{24}}}}\)
    \item \(\sqrt[3]{{9{x^2}}}\,\sqrt[3]{{6{x^2}}}\)
    \end{enumerate}
  \end{multicols}
\end{oefening}

\begin{oefening}
  Vereenvoudig (schrijf dus alle machtswortels in hun vereenvoudigde wortelvorm):
  \begin{multicols}{2}
    \begin{enumerate}[(a)]
      \itemsep1em
    \item \( \displaystyle \frac{4}{{\sqrt x }}\)
    \item \( \displaystyle \sqrt[5]{{\frac{2}{{{x^3}}}}}\)
    \item \( \displaystyle \frac{1}{{3 - \sqrt x }}\)
    \item \( \displaystyle \frac{5}{{4\sqrt x + \sqrt 3 }}\)
    \end{enumerate}
  \end{multicols}
\end{oefening}

\subsection{Correcte eigenschap $\sqrt[n]{a^n}$}

In bovenstaande eigenschappen hebben we geëist dat het grondtal $a$ positief moest zijn. Dit om ons werk in dit hoofdstuk gemakkelijker te maken. Dit hoeft echter niet, hier is de algemene regel:

\[\sqrt[n]{{{a^n}}} = \left\{ {\begin{array}{*{20}{l}}{\left| a \right|}&{{\mbox{if }}n{\mbox{ is
even}}}\\a&{{\mbox{if }}n{\mbox{ is odd}}}\end{array}} \right.\]

waarbij \(\left| a \right|\) de absolute waarde is van $a$. De uitkomst is dus altijd positief als de index en exponent $n$ even zijn.

Een voorbeeld:

\[\sqrt[8]{{{x^8}}} = \left| x
\right|\hspace{0.25in}\hspace{0.25in}{\mbox{AND}}\hspace{0.25in}\hspace{0.25in}\,\,\,\,\,\,\sqrt[{11}]{{{x^
{11}}}} = x\]

Voor vierkantswortels wordt dit

\[\sqrt {{x^2}} = \left| x \right|\]

\begin{oefening}
  Schrijf volgende uitdrukkingen als macht
  \begin{multicols}{2}
    \begin{enumerate}[(a)]
      \itemsep1em
    \item \(\sqrt[7]{y}\)
    \item \(\sqrt[3]{{{x^2}}}\)
    \item \(\sqrt[6]{{ab}}\)
    \item \(\sqrt {{w^2}{v^3}} \)
    \end{enumerate}
  \end{multicols}
\end{oefening}

\begin{oefening}
  Bereken zonder \zrm{ZRM}
  \begin{multicols}{3}
    \begin{enumerate}[(a)]
      \itemsep1em
    \item  \(\sqrt[4]{{81}}\)
    \item \(\sqrt[3]{{ - 512}}\)
    \item \(\sqrt[3]{{1000}}\)
    \end{enumerate}
  \end{multicols}
\end{oefening}

\begin{oefening}
  Vereenvoudig, veronderstel dat $x$, $y$ en $z$ positief zijn
  \begin{multicols}{3}
    \begin{enumerate}[(a)]
      \itemsep1em
    \item \(\sqrt[3]{{{x^8}}}\)
    \item \(\sqrt {8{y^3}} \)
    \item \(\sqrt[4]{{{x^7}{y^{20}}{z^{11}}}}\)
    \item \(\sqrt[3]{{54{x^6}{y^7}{z^2}}}\)
    \item \(\sqrt[4]{{4{x^3}y}}\,\,\sqrt[4]{{8{x^2}{y^3}{z^5}}}\)
    \end{enumerate}
  \end{multicols}
\end{oefening}

\begin{oefening}
  Vereenvoudig, veronderstel dat $x$ positief is
  \begin{enumerate}[(a)]
    \itemsep1em
  \item \(\sqrt x \left( {4 - 3\sqrt x } \right)\)
  \item \(\left( {2\sqrt x + 1} \right)\left( {3 - 4\sqrt x } \right)\)
\item \(\left( {\sqrt[3]{x} + 2\,\,\sqrt[3]{{{x^2}}}} \right)\left( {4 - \sqrt[3]{{{x^2}}}} \right)\)
  \end{enumerate}
\end{oefening}

\begin{oefening}
  Vereenvoudig, veronderstel dat $x$ en $y$ positief zijn
  \begin{multicols}{2}
    \begin{enumerate}[(a)]
      \itemsep1em
    \item \(\displaystyle \frac{6}{{\sqrt x }}\)
    \item \(\displaystyle \frac{9}{{\sqrt[3]{{2x}}}}\)
    \item \(\displaystyle \frac{4}{{\sqrt x + 2\sqrt y }}\)
    \item \(\displaystyle \frac{{10}}{{3 - 5\sqrt x }}\)
    \end{enumerate}
  \end{multicols}
\end{oefening}

\needspace{5cm}
\subsection{Extra oefeningen}

\begin{oefening}
Geef de definitie van een $n$-de machtswortel. ($b$ is een $n$-de machtswortel uit $a$ \ldots)
\end{oefening}

\begin{oefening}
Bespreek het bestaan van machtswortels. ($n$ al dan niet even, $a \lesseqgtr 0$)
\end{oefening}

\begin{oefening}*
Leid mbv. de definitie de volgende eigenschappen van machtswortels af:
\begin{align*}
  \sqrt[n]{ab} &= \sqrt[n]{a}\sqrt[n]{b}\\
  \sqrt[n]{\dfrac{a}{b}} &= \dfrac{\sqrt[n]{a}}{\sqrt[n]{b}}\\
  \sqrt[n]{a^m} &= \left(\sqrt[n]{a}\right)^m\\
  \sqrt[n]{\sqrt[m]{a}} &= \sqrt[n\cdot m]{a}\\
  \sqrt[n\cdot p]{a^{m\cdot p}} &= \sqrt[n]{a^m}\\
  \sqrt[n]{a^n} &= a
\end{align*}
\end{oefening}

\begin{oefening}
  Schrijf volgende uitdrukkingen als macht
  \begin{multicols}{3}
    \begin{enumerate}[(a)]
      \itemsep1em
    \item \(\sqrt {3n} \)
    \item \(\sqrt[6]{{2y}}\)
    \item \(\sqrt[5]{{7{x^3}}}\)
    \item \(\sqrt[4]{{xyz}}\)
    \item \(\sqrt {x + y} \)
    \item \(\sqrt[3]{{{a^3} + {b^3}}}\)
    \end{enumerate}
  \end{multicols}
\end{oefening}

\begin{oefening}
  Bereken zonder \zrm{ZRM}
  \begin{multicols}{3}
    \begin{enumerate}[(a)]
      \itemsep1em
    \item \(\sqrt {256} \)
    \item \(\sqrt[4]{{256}}\)
    \item \(\sqrt[8]{{256}}\)
    \item \(\sqrt[5]{{ - 1024}}\)
    \item \(\sqrt[3]{{ - 216}}\)
    \item \(\sqrt[3]{{343}}\)
    \end{enumerate}
  \end{multicols}
\end{oefening}

\begin{oefening}
  Vereenvoudig, veronderstel dat $x$, $y$ en $z$ positief zijn
  \begin{multicols}{3}
    \begin{enumerate}[(a)]
      \itemsep1em
    \item \(\sqrt {{z^5}} \)
    \item \(\sqrt[3]{{{z^5}}}\)
    \item \(\sqrt[3]{{16{x^{17}}}}\)
    \item \(\sqrt[6]{{128{y^{11}}}}\)
    \item \(\sqrt {{x^3}{y^{17}}{z^4}} \)
    \item \(\sqrt[4]{{{x^3}{y^{20}}{z^5}}}\)
    \item \(\sqrt[4]{{729{x^7}y\,{z^{13}}}}\)
    \item \(\sqrt[3]{{4{x^2}y}}\,\,\,\sqrt[3]{{10{x^5}{y^2}}}\)
    \item \(\sqrt {3x} \,\,\sqrt {6x} \,\,\sqrt {14x} \)
    \item \(\sqrt[4]{{2x{y^3}}}\,\,\,\sqrt[4]{{32{x^2}{y^2}}}\)
    \end{enumerate}
  \end{multicols}
\end{oefening}

\begin{oefening}
  Vereenvoudig, veronderstel dat $x$ positief is
  \begin{multicols}{2}
    \begin{enumerate}[(a)]
      \itemsep1em
    \item \(\left( {2\sqrt x + 4} \right)\left( {\sqrt x - 7} \right)\)
    \item \(\sqrt[3]{x}\left( {\sqrt[3]{x} + 2\sqrt[3]{{{x^4}}}} \right)\)
    \item \(\left( {\sqrt x + \sqrt {2y} } \right)\left( {\sqrt x - \sqrt {2y} } \right)\)
    \item \({\left( {\sqrt[4]{x} + \sqrt[4]{{{x^2}}}} \right)^2}\)
    \end{enumerate}
  \end{multicols}
\end{oefening}

\begin{oefening}
  Vereenvoudig, veronderstel dat $x$ en $y$ positief zijn
  \begin{multicols}{3}
    \begin{enumerate}[(a)]
      \itemsep1em
    \item \(\displaystyle \frac{9}{{\sqrt y }}\)
    \item \(\displaystyle \frac{3}{{\sqrt {7x} }}\)
    \item \(\displaystyle \frac{1}{{\sqrt[4]{x}}}\)
    \item \(\displaystyle \frac{{12}}{{\sqrt[5]{{3{x^2}}}}}\)
    \item \(\displaystyle \frac{2}{{4 - \sqrt x }}\)
    \item \(\displaystyle \frac{9}{{\sqrt {3y} + 2}}\)
    \item \(\displaystyle \frac{4}{{\sqrt 7 - 6\sqrt x }}\)
    \item \(\displaystyle \frac{{ - 6}}{{\sqrt {5x} + 10\sqrt y }}\)
    \item \(\displaystyle \frac{{4 + x}}{{x - \sqrt x }}\)
    \end{enumerate}
  \end{multicols}
\end{oefening}

\begin{oefening}
  Zijn volgende beweringen waar of vals?
  \begin{enumerate}[(a)]
    \itemsep1em
  \item $3{x^{\frac{1}{2}}} = \sqrt {3x}$ \hfill \texttt{WAAR / VALS}\mbox{\hspace{5cm}}
  \item $\sqrt[3]{{x + 6}} = \sqrt[3]{x} + \sqrt[3]{6}$ \hfill \texttt{WAAR / VALS}\mbox{\hspace{5cm}}
  \item $\sqrt[4]{{{x^2}}} = \sqrt x$ \hfill \texttt{WAAR / VALS}\mbox{\hspace{5cm}}
  \end{enumerate}
\end{oefening}

\begin{oefening}*
Toon aan dat
$$\sqrt{2-\sqrt{2+\sqrt{2+\sqrt{3}}}}\cdot\sqrt{2+\sqrt{2+\sqrt{2+\sqrt{3}}}}\cdot\sqrt{2+\sqrt{2+\sqrt{3}}}\cdot\sqrt{2+\sqrt{3}} = 1\;.$$
\end{oefening}

\begin{oefening}
Bereken zonder \zrm{ZRM}
$$\sqrt{1+101\cdot99}$$
{\em Hint: merkwaardig product}
\end{oefening}

\begin{oefening}{\scriptsize Bron: Cambridge interview problem}\\
  Zonder een \zrm{ZRM} te gebruiken, aan wat is  $\displaystyle\sqrt{3 - 2\sqrt{2}}$  gelijk
  \begin{enumerate}[(A)]
  \item $\sqrt{3}-1$
  \item $\sqrt{2}-1$
  \item $\sqrt{3}-\sqrt{2}$
  \end{enumerate}
  {\em Hint: Herschrijf de uitdrukking onder de wortel als een kwadraat door een merkwaardig product te gebruiken.}
\end{oefening}

\pagebreak
\section{Exponentiële vergelijkingen}

\begin{oefening}
Los op in $\mathbb{R}$.
\begin{exlist}{3}
  \item $5^{3x}=5^{7x-2}$
  \item $4^{t^2}=4^{6-t}$
  \item $3^z=9^{z+5}$
  \item $4^{5-9x}=\dfrac{1}{8^{x-2}}$
  \item $2^{4x}=8^{x+1}$
  \item $2^{4x}=16^{x+1}$
\end{exlist}
\end{oefening}


\begin{oefening}
Los op in $\mathbb{R}$, geef telkens de oplossingenverzameling en maak de proef.
\begin{multicols}{3}
\begin{enumerate}[(a)]
  \itemsep.5em
  \item $7^x=49$
  \item $4^x=1$
  \item $6^{3x}=36$
  \item $8^{3x}=-7$
  \item $8^x=2^{x-1}$
  \item $0.5^{4x}=\dfrac{1}{2}$
  \item $36^{x-1}=\sqrt{6}$
  \item $5^{3x-2}=25^x$
  \item $2^{5x}=16^{x+1}$
\end{enumerate}
\end{multicols}
\end{oefening}

\begin{oefening}
Los op in $\mathbb{R}$, geef telkens de oplossingenverzameling en maak de proef.
\begin{multicols}{3}
\begin{enumerate}[(a)]
  \itemsep.5em
  \item $\left(2^{x+1}\right)^2=1024$
  \item $2^{x+1}+2^{x-2}=72$
  \item $9^{x+1}-28\cdot 3^x=-3$
\end{enumerate}
\end{multicols}
\end{oefening}

\begin{oefening}*
Los op in $\mathbb{R}$ door telkens een gepaste substitutie te maken, geef telkens de oplossingenverzameling en maak de proef.
\begin{multicols}{2}
\begin{enumerate}[(a)]
  \item $3^{2x}+3^x-12=0$
  \item $2^{2x}-2^x-12=0$
  \item $4^{2x}-17\cdot 4^x + 16=0$
  \item $4^x+3\cdot 2^x+2=0$
  \item $5^{2x}-30\cdot 5^x+125=0$
  \item $4^x-5\cdot 2^{x+1}+16=0$
\end{enumerate}
\end{multicols}
\end{oefening}

\begin{oefening}
We kunnen exponentiële vergelijkingen ook grafisch oplossen. We doen dit bijvoorbeeld voor de vergelijking
$$2^{x-2}=0.5^{x-2}\;.$$
\begin{enumerate}[(a)]
  \item Los deze eerst algebraïsch op in $\mathbb{R}$.\\
        {\em Hint: zorg dat de macht in het rechterlid grondtal 2 krijgt.}
  \item Maak van het linkerlid de exponentiële functie $f(x)=2^{x-1}$ en teken de grafiek.
  \item Maak ook van het rechterlid een functie, noem deze $g(x)$, en teken de grafiek op hetzelfde assenstelsel.
  \item De $x$-waarden van de snijpunten van de grafieken zijn oplossingen, zoek de oplossingen en vergelijk deze met (a).
\end{enumerate}
\end{oefening}

\begin{oefening}*
Los volgende exponentiële vergelijking algebraïsch op, zonder gebruik te maken van substitutie.
$$4^x+2=3\cdot2^x$$
{\em Hint: $a\cdot b=0$ als $a=0$ of $b=0$.}
\end{oefening}

\pagebreak
\section{Exponentiële functies}

\begin{oefening}
Maak de grafiek van de volgende exponentiële functies
\begin{multicols}{3}
\begin{enumerate}[(a)]
  \itemsep.5em
  \item $f(x)=10^x$
  \item $f(x)=0.5^x$
  \item $f(x)=e^x$
\end{enumerate}
\end{multicols}
Het getal $e$, het getal van Euler, is een constante die je op je rekentoestel vindt. Druk \zrm{SHIFT} \zrm{ln} en dan de exponent.
\end{oefening}

\begin{oefening}
Gegeven de grafiek van de exponentiële functie $f(x)=a^x$. Bepaal telkens de waarde van $a$.
\begin{multicols}{3}
\begin{enumerate}[(a)]
  \item\raisebox{-3cm}{
\definecolor{cqcqcq}{rgb}{0.75,0.75,0.75}
\begin{tikzpicture}[line cap=round,line join=round,>=triangle 45,x=1.0cm,y=1.0cm]
\draw [color=cqcqcq,dash pattern=on 1pt off 1pt, xstep=0.5cm,ystep=0.5cm] (-1.06,-0.3) grid (2.68,2.58);
\draw[->,color=black] (-1.06,0) -- (2.68,0);
\foreach \x in {1,2}
\draw[shift={(\x,0)},color=black] (0pt,2pt) -- (0pt,-2pt) node[below] {\footnotesize $\x$};
\draw[->,color=black] (0,-0.3) -- (0,2.58);
\foreach \y in {1,2}
\draw[shift={(0,\y)},color=black] (2pt,0pt) -- (-2pt,0pt) node[left] {\footnotesize $\y$};
\draw[color=black] (0pt,-10pt) node[right] {\footnotesize $0$};
\clip(-1.06,-0.3) rectangle (2.68,2.58);
\draw(2.02,-1.54) -- (2.95,-1.54);
\draw[line width=1.6pt, smooth,samples=100,domain=-1.06:2.68] plot(\x,{exp(ln(0.25)*\x});
\end{tikzpicture}
}
  \item\raisebox{-3cm}{
\definecolor{cqcqcq}{rgb}{0.75,0.75,0.75}
\begin{tikzpicture}[line cap=round,line join=round,>=triangle 45,x=1.0cm,y=1.0cm]
\draw [color=cqcqcq,dash pattern=on 1pt off 1pt, xstep=0.5cm,ystep=0.5cm] (-1.06,-0.3) grid (2.68,2.58);
\draw[->,color=black] (-1.06,0) -- (2.68,0);
\foreach \x in {1,2}
\draw[shift={(\x,0)},color=black] (0pt,2pt) -- (0pt,-2pt) node[below] {\footnotesize $\x$};
\draw[->,color=black] (0,-0.3) -- (0,2.58);
\foreach \y in {1,2}
\draw[shift={(0,\y)},color=black] (2pt,0pt) -- (-2pt,0pt) node[left] {\footnotesize $\y$};
\draw[color=black] (0pt,-10pt) node[right] {\footnotesize $0$};
\clip(-1.06,-0.3) rectangle (2.68,2.58);
\draw(2.02,-1.54) -- (2.95,-1.54);
\draw[line width=1.6pt, smooth,samples=100,domain=-1.06:2.68] plot(\x,{exp(ln(1.5)*\x});
\end{tikzpicture}
}
  \item\raisebox{-3cm}{
\definecolor{cqcqcq}{rgb}{0.75,0.75,0.75}
\begin{tikzpicture}[line cap=round,line join=round,>=triangle 45,x=1.0cm,y=1.0cm]
\draw [color=cqcqcq,dash pattern=on 1pt off 1pt, xstep=0.5cm,ystep=0.5cm] (-1.06,-0.3) grid (2.68,2.58);
\draw[->,color=black] (-1.06,0) -- (2.68,0);
\foreach \x in {1,2}
\draw[shift={(\x,0)},color=black] (0pt,2pt) -- (0pt,-2pt) node[below] {\footnotesize $\x$};
\draw[->,color=black] (0,-0.3) -- (0,2.58);
\foreach \y in {1,2}
\draw[shift={(0,\y)},color=black] (2pt,0pt) -- (-2pt,0pt) node[left] {\footnotesize $\y$};
\draw[color=black] (0pt,-10pt) node[right] {\footnotesize $0$};
\clip(-1.06,-0.3) rectangle (2.68,2.58);
\draw(2.02,-1.54) -- (2.95,-1.54);
\draw[line width=1.6pt, smooth,samples=100,domain=-1.06:2.68] plot(\x,{exp(ln(2)*\x});
\end{tikzpicture}
}
\end{enumerate}
\end{multicols}
\end{oefening}

\begin{oefening}
Maak de grafiek van de volgende functies:
\begin{multicols}{3}
\begin{enumerate}[(a)]
  \itemsep.5em
  \item $f(x)=2^x+2.5$
  \item $f(x)=2^{x-1.5}$
  \item $f(x)=0.5^{2x}$
\end{enumerate}
\end{multicols}
Hint: Probeer (c) eerst in de vorm $f(x)=a^x$ te brengen.
\end{oefening}

\begin{oefening}
Bepaal de volgende limieten:
\begin{multicols}{3}
\begin{enumerate}[(a)]
  \itemsep.5em
  \item $\displaystyle\lim_{x\to+\infty}3^x$
  \item $\displaystyle\lim_{x\to-\infty}3^x$
  \item $\displaystyle\lim_{x\to0}3^x$
  \item $\displaystyle\lim_{x\to1}3^x$
  \item $\displaystyle\lim_{x\to+\infty}0.01^x$
  \item $\displaystyle\lim_{x\to-\infty}1.1^x$
  \item $\displaystyle\lim_{x\to-\infty}0.99^x$
  \item $\displaystyle\lim_{x\to+\infty}0.5^x+1.5$
  \item $\displaystyle\lim_{x\to+\infty}2^{-x}$
\end{enumerate}
\end{multicols}
\end{oefening}

\begin{oefening}
  % source: Dit is een afschatting die ik nodig had tijdens een bewijs uit "Divide and Conquer, Sorting and Searching, and Randomized Algorithms" by Stanford University op Coursera
  Toon m.b.v. een grafiek aan dat
  $$1+x \leq e^x$$
\end{oefening}

\begin{oefening}*
Gegeven de grafiek van $f(x)=e^x$, schets de grafieken van de functies met voorschriften:
\begin{multicols}{3}
\begin{enumerate}[(a)]
  \itemsep.5em
  \item $g(x)=f(x)-1$
  \item $g(x)=-f(x)$
  \item $g(x)=2\cdot f(x)$
  \item $g(x)=-2\cdot f(x)$
  \item $g(x)=f(x-2.5)$
  \item $g(x)=f(-x)$
\end{enumerate}
\end{multicols}
\end{oefening}

\begin{oefening}*
Een functie met voorschrift $y=k\cdot a^x$ gaat door $P(-1,2)$ en $Q(2,\dfrac{125}{32})$. Bepaal $a$ en $k$.
\end{oefening}

\pagebreak
\section{Toepassingen op exponentiële functies}

\begin{oefening}
In een bepaald land neemt de bevolking jaarlijks met $2\%$ toe. Momenteel wonen er 500 000 mensen. Hoeveel inwoners zijn er over 10 jaar?
\end{oefening}

\begin{oefening}
  Je erft een stuk land van je suikernonkel. Hij kocht dat stuk land in 1995 voor $600000$ Belgische frank (bij de overgang naar euro was 1 euro = 40 Belgische frank). De waarde van het land stijgt ongeveer met $5\%$ per jaar.
  \begin{enumerate}[(a)]
  \item Moest de euro al bestaan hebben. Hoeveel zou je nonkel toen hebben uitgegeven in euro.
  \item Tijdens de eeuwwisseling twijfelde je nonkel om het te verkopen. Hoeveel zou hij toen gekregen hebben voor het stuk land.
  \item Als jij het nu zou verkopen, hoeveel krijg je dan?
  \end{enumerate}
\end{oefening}

\begin{oefening}\\[-1.5cm]
  \begin{minipage}{0.6\linewidth}
    Een amateur renner neemt doping bij het begin van de Ronde van Vlaanderen voor amateurs. Elk uur vermindert de hoeveelheid doping met $18\%$. De start was om 10u30 en na $270 \km$ komt de renner aan in Oudenaarde om 17u41.
  \end{minipage}
  \begin{minipage}{0.39\linewidth}
    \vspace{-0.5cm}
    \includegraphics[width=\textwidth]{rvv-logo}
  \end{minipage}
  \vspace{-1.5cm}
  \begin{enumerate}[(a)]
  \item Hoe lang was de renner onderweg?
  \item Hoe snel reed de renner gemiddeld?
  \item Hoeveel procent van de doping was nog aanwezig in zijn bloed bij aankomst?
  \item Stel dat renner initieel $20 \ml$ van het product nam, na hoeveel tijd moet hij opnieuw doping nemen zodat er steeds op zijn minst $5 \ml$ van het product aanwezig blijft.
  \end{enumerate}
\end{oefening}

\begin{oefening}
  Als je ademt, dan wordt ongeveer $12\%$ van de lucht in de longen vervangen na één ademhaling.
  \begin{enumerate}[(a)]
  \item Geef een functie die de exponentiële daling modelleert als de hoeveelheid originele lucht over in de longen $500 \ml$ is.
  \item Hoeveel van de originele lucht is over na 200 ademhalingen?
  \end{enumerate}
\end{oefening}

\begin{oefening}
  Een volwassen vrouw neemt $400 \mg$ ibuprofen. Elk uur vermindert de hoeveelheid ibuprofen in de vrouw haar bloed met $26\%$. Hoeveel ibuprofen is er over na 6 uur.
\end{oefening}

\begin{oefening}
In een bedrijf wordt de waarde $P$ in euro van een productiemachine $t$ jaar na aankoop geschat op:
$$P = 20000 e^{-0.5\cdot t}\;.$$
\begin{enumerate}[(a)]
  \item Bereken de aankoopprijs.
  \item Bereken de waarde 4 jaar na aankoop.
\end{enumerate}
\end{oefening}

\begin{oefening}
De luchtdruk $P$ op een hoogte van $h$ kilometer wordt gegeven door:
$$P = P_0\cdot e^{-0.12\cdot h}$$
met $P_0$ de luchtdruk op zeeniveau. Welk percentage van $P_0$ heb je op:
\begin{enumerate}[(a)]
  \item een skipiste van 2500 m hoogte?
  \item de Mont Blanc (4810 m), de hoogste berg van Europa?
  \item de Everest (8848 m), de hoogste berg op aarde?
  \item een hoogte van 160 km, de minimale hoogte voor een satelliet?
\end{enumerate}
\end{oefening}

\begin{oefening}
De ziekte cholera wordt veroorzaakt door de cholera bacterie. Is $N_0$ het aantal bacteriën op het tijdstip 0, dan wordt het aantal bacteriën $N$ na $t$ uur gegeven door:
$$N=N_0\cdot e^{1.386 t}\;.$$
Veronderstel dat er bij aanvang 20 bacteriën zijn. Hoeveel zijn er na 1 uur, na 2 uur, na 5 uur, na 1 dag en na één week?
\end{oefening}

\begin{oefening}*
In een school met $n$ leerlingen wordt het aantal leerlingen $n'$ die een gerucht gehoord hebben, $t$ dagen na het ontstaan, gegeven door de formule $(t\geq 1)$:
$$n' = n(1-e^{-0.12\cdot t})\;.$$
Hoeveel percent van de leerlingen kent het gerucht na 1 dag, hoeveel na 8 dagen en hoeveel na 30 dagen?
\end{oefening}

\appendix
\cleardoublepage
\section{Machten met natuurlijk grondtal en natuurlijke exponent}

Volgende machten kennen we nog uit het hoofd:

\begin{adjustwidth}{-1cm}{1cm}
\begin{center}
\begin{tabular}{|c|c|c|c|c|c|c|c|c|c|}
  \hline
  $a^n$ & $2$ & $3$ & $4$ & $5$ & $6$ & $7$ & $8$ & $9$ & $10$\\
  \hline
  2     & 4   & 8   & 16  & 32  & 64  & 128 & 256 & 512 & 1024\\
  \cline{1-10}
  3     & 9   & 27  & 81  & 243 & 729\\
  \cline{1-6}
  4     & 16  & 64  & 256 & 1024\\
  \cline{1-5}
  5     & 25  & 125 & 625\\
  \cline{1-4}
  6     & 36  & 216\\
  \cline{1-3}
  7     & 49  & 343\\
  \cline{1-3}
  8     & 64  & 512\\
  \cline{1-3}
  9     & 81  & 729\\
  \cline{1-6}
  10    & 100  & 1\,000 & 10\,000 & 100\,000 & 1\,000\,000\\
  \cline{1-6}
  11    & 121\\
  \cline{1-2}
  12    & 144\\
  \cline{1-2}
  13    & 169\\
  \cline{1-2}
  14    & 196\\
  \cline{1-2}
  15    & 225\\
  \cline{1-2}
  16    & 256\\
  \cline{1-2}
  17    & 289\\
  \cline{1-2}
  18    & 324\\
  \cline{1-2}
  19    & 361\\
  \cline{1-2}
  20    & 400\\
  \cline{1-2}
\end{tabular}
\end{center}
\end{adjustwidth}

\end{document}
