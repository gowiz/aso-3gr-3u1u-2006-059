\documentclass[12pt,twoside,a4paper]{article}

\input{../gowiz.tex}

\newenvironment{definitie}
{
  \vspace{0.4cm}
  \begin{mdframed}[nobreak=true,frametitle={Definitie}]
  }{%
  \end{mdframed}
}

\newenvironment{eigenschap}
{
  \vspace{0.4cm}
  \begin{mdframed}[nobreak=true,frametitle={Eigenschap}]
  }{%
  \end{mdframed}
}

\begin{document}

\begin{center}
  \begin{mdframed}
    \centering
    \fontsize{50}{60}\selectfont Limieten
  \end{mdframed}
  \vfill
  \includegraphics[width=\textwidth]{limieten}
  \vfill
\end{center}

\subsection*{Doelstelling}
Je \hfill  {\scriptsize(LP 2006-059, LI 1.4)}
\begin{itemize}
\item kent het begrip limiet en kan grafisch limieten bepalen
\item kan met behulp van rekenregels limieten berekenen van veeltermfuncties en rationale functies
\item kan met behulp van limieten de horizontale en verticale asymptoten van rationale functies bepalen
\item kan de schuine asymptoten van rationale functies bepalen (U)
\end{itemize}

\thispagestyle{empty}
\newpage

\tableofcontents
\thispagestyle{empty}
\newpage

\pagenumbering{arabic}

\pagestyle{fancy}
\fancyhead[RO,LE]{Limieten}
\fancyhead[RE,LO]{}

\cleardoublepage
\section{Begrip limiet}

\subsection{Limieten in een reëel getal}

\subsubsection*{Eindige limieten in een reëel getal}
\begin{definitie}
We zeggen dat een reëel getal $b$ de {\bf limiet is van $f$ voor $x$ gaande naar $a$} als de functiewaarden $f(x)$ willekeurig dicht bij $b$ komen voor $x$-waarden in de buurt van $a$. We noteren dit als
  $$\lim_{x\to a} f(x)=b$$
\end{definitie}

In andere gevallen spreken we af dat de limiet in $a$ niet bestaat.

{\em Maak de oefeningen op aparte bladen.}

\begin{oefening}
  Beschouw de functie
  $$f:\mathbb{R}\backslash\{1\} \to \mathbb{R}:x \mapsto x^2-2x+2$$
  Schets deze functie, merk op dat het domein het getal $1$ niet bevat. Bepaal grafisch de limieten
  \begin{enumerate}[(a)]
  \itemsep.5em
  \item $\displaystyle\lim_{x\to0} f(x)$
  \item $\displaystyle\lim_{x\to1} f(x)$
  \end{enumerate}
\end{oefening}

\begin{oefening}
  Beschouw de functie
  $$f(x)=\dfrac{x^3-x^2-14x+24}{x-2}$$
  Schets deze functie, merk op dat het domein het getal $2$ niet bevat. Bepaal grafisch de limieten
  \begin{enumerate}[(a)]
  \itemsep.5em
  \item $\displaystyle\lim_{x\to1} f(x)$
  \item $\displaystyle\lim_{x\to2} f(x)$
  \end{enumerate}
\end{oefening}

\begin{oefening}
  Beschouw de volgende grafiek van de een functie $f$
  \begin{center}
    \definecolor{cqcqcq}{rgb}{0.75,0.75,0.75}
    \begin{tikzpicture}[scale=1.5,line cap=round,line join=round,>=triangle 45,x=1.0cm,y=1.0cm]
      \draw [color=cqcqcq,dash pattern=on 1pt off 1pt, xstep=1.0cm,ystep=1.0cm] (-3.2,-1.38) grid (2.38,3.18);
      \draw[->,color=black] (-3.2,0) -- (2.38,0);
      \foreach \x in {-3,-2,-1,1,2}
      \draw[shift={(\x,0)},color=black] (0pt,2pt) -- (0pt,-2pt) node[below] {\footnotesize $\x$};
      \draw[->,color=black] (0,-1.38) -- (0,3.18);
      \foreach \y in {-1,1,2,3}
      \draw[shift={(0,\y)},color=black] (2pt,0pt) -- (-2pt,0pt) node[left] {\footnotesize $\y$};
      \draw[color=black] (0pt,-10pt) node[right] {\footnotesize $0$};
      \clip(-3.2,-1.38) rectangle (2.38,3.18);
      \draw[line width=1.6pt, smooth,samples=100,domain=-3.2:-1.1] plot(\x,{(((\x)+1)^2+2)});
      \draw[line width=1.6pt, smooth,samples=100,domain=-0.9:2.38] plot(\x,{(-((\x)+1)^2+1)});
      \begin{scriptsize}
        \draw [color=black] (-1,2) circle (2.0pt);
        \draw [color=black] (-1,1) circle (2.0pt);
      \end{scriptsize}
    \end{tikzpicture}
  \end{center}
  Bepaal grafisch de limieten
  \begin{enumerate}[(a)]
  \itemsep.5em
  \item $\displaystyle\lim_{x\to-2} f(x)$
  \item $\displaystyle\lim_{x\to-1} f(x)$
  \item $\displaystyle\lim_{x\to0} f(x)$
  \end{enumerate}
\end{oefening}

\needspace{3cm}
\subsubsection*{Oneindige limieten in een reëel getal}

\begin{definitie}
  We zeggen de {\bf limiet van $f$ voor $x$ gaande naar $a$ is $+\infty$} als de functiewaarden $f(x)$ willekeurig groot worden voor $x$-waarden in de buurt van $a$. We noteren dit als
  $$\lim_{x\to a}f(x)=+\infty$$
  Op analoge wijze definiëren en noteren we $\displaystyle\lim_{x\to a}f(x)=-\infty$
\end{definitie}

\begin{oefening}
  Beschouw hieronder de grafiek van $f:\mathbb{R}_0\to\mathbb{R}:x\mapsto\dfrac{1}{x^2}$.
  \begin{center}
    \definecolor{cqcqcq}{rgb}{0.75,0.75,0.75}
    \begin{tikzpicture}[yscale=0.5,line cap=round,line join=round,>=triangle 45,x=1.0cm,y=1.0cm]
      \draw [color=cqcqcq,dash pattern=on 2pt off 2pt, xstep=1.0cm,ystep=2.0cm] (-5,-1.41) grid (5,10.23);
      \draw[->,color=black] (-5.15,0) -- (5.28,0);
      \foreach \x in {-5,-4,-3,-2,-1,1,2,3,4,5}
      \draw[shift={(\x,0)},color=black] (0pt,2pt) -- (0pt,-2pt) node[below] {\footnotesize $\x$};
      \draw[->,color=black] (0,-1.41) -- (0,10.23);
      \foreach \y in {,2,4,6,8,10}
      \draw[shift={(0,\y)},color=black] (2pt,0pt) -- (-2pt,0pt) node[left] {\footnotesize $\y$};
      \draw[color=black] (0pt,-10pt) node[right] {\footnotesize $0$};
      \clip(-5.15,-1.41) rectangle (5.28,10.23);
      \draw[line width=1.6pt, smooth,samples=100,domain=-5:5] plot(\x,{1/(\x)^2});
    \end{tikzpicture}
  \end{center}
  Bepaal de limieten
  \begin{multicols}{2}
  \begin{enumerate}[(a)]
  \itemsep.5em
  \item $\displaystyle\lim_{x\to-1} f(x)$
  \item $\displaystyle\lim_{x\to0} f(x)$
  \item $\displaystyle\lim_{x\to2} f(x)$
  \item $\displaystyle\lim_{x\to10000} f(x)$
  \end{enumerate}
  \end{multicols}
\end{oefening}

\subsection{Linker- en rechterlimieten in een reëel getal}

Limieten van functies in een reëel getal $a$ (zowel eindige limieten als limieten $\pm\infty$) worden ook gedefinieerd in gevallen waar de functiewaarden een bepaald gedrag vertonen alleen voor $x$-waarden links of rechts van $a$. We spreken dan van linkerlimiet en rechterlimiet.

\begin{definitie}
  We zeggen dat een reëel getal b de {\bf linkerlimiet is van $f$ voor $x$ gaande naar $a$} als de functiewaarden $f(x)$ willekeurig dicht bij $b$ komen voor $x$-waarden in de buurt van $a$ en links van $a$ genomen. We noteren dit als
  $$\lim_{x\underset{<}{\to}a}f(x)=b$$
\end{definitie}

\begin{definitie}
  We zeggen dat een reëel getal b de {\bf rechterlimiet is van $f$ voor $x$ gaande naar $a$} als de functiewaarden $f(x)$ willekeurig dicht bij $b$ komen voor $x$-waarden in de buurt van $a$ en rechts van $a$ genomen. We noteren dit als
  $$\lim_{x\underset{>}{\to}a}f(x)=b$$
\end{definitie}

\begin{oefening}
  Beschouw de functie
  $$f:\mathbb{R}\to\mathbb{R}:x\mapsto\begin{cases}
    1 & \mbox{ als } x > 2\\
    -1& \mbox{ als } x < 2
  \end{cases}
  $$
  \begin{enumerate}[(a)]
  \item Wat is het domein?
  \item Teken de grafiek.
  \item Bepaal de linker- en rechterlimiet voor $x$ gaande naar $2$.
  \end{enumerate}
\end{oefening}

\begin{oefening}
  Bepaal nogmaals de limiet $\displaystyle\lim_{x\to-1} f(x)$ van de volgende functie
  \begin{center}
    \definecolor{cqcqcq}{rgb}{0.75,0.75,0.75}
    \begin{tikzpicture}[scale=1.5,line cap=round,line join=round,>=triangle 45,x=1.0cm,y=1.0cm]
      \draw [color=cqcqcq,dash pattern=on 1pt off 1pt, xstep=1.0cm,ystep=1.0cm] (-3.2,-1.38) grid (2.38,3.18);
      \draw[->,color=black] (-3.2,0) -- (2.38,0);
      \foreach \x in {-3,-2,-1,1,2}
      \draw[shift={(\x,0)},color=black] (0pt,2pt) -- (0pt,-2pt) node[below] {\footnotesize $\x$};
      \draw[->,color=black] (0,-1.38) -- (0,3.18);
      \foreach \y in {-1,1,2,3}
      \draw[shift={(0,\y)},color=black] (2pt,0pt) -- (-2pt,0pt) node[left] {\footnotesize $\y$};
      \draw[color=black] (0pt,-10pt) node[right] {\footnotesize $0$};
      \clip(-3.2,-1.38) rectangle (2.38,3.18);
      \draw[line width=1.6pt, smooth,samples=100,domain=-3.2:-1.1] plot(\x,{(((\x)+1)^2+2)});
      \draw[line width=1.6pt, smooth,samples=100,domain=-0.9:2.38] plot(\x,{(-((\x)+1)^2+1)});
      \begin{scriptsize}
        \draw [color=black] (-1,2) circle (2.0pt);
        \draw [color=black] (-1,1) circle (2.0pt);
      \end{scriptsize}
    \end{tikzpicture}
  \end{center}
  Gebruik nu echter linker- en rechterlimieten.
\end{oefening}

Ook oneindige linker- en rechterlimieten in een getal worden gedefinieerd. Bijvoorbeeld voor linkerlimieten wordt dit:

\begin{definitie}
  We zeggen de {\bf linkerlimiet is van $f$ voor $x$ gaande naar $a$ is $+\infty$} als de functiewaarden $f(x)$ willekeurig groot worden voor $x$-waarden in de buurt van $a$ en links van $a$ genomen. We noteren dit als
  $$\lim_{x\underset{<}{\to}a}f(x)=+\infty$$
\end{definitie}

\begin{oefening}
  Definieer zelf de oneindige rechterlimiet in een getal.
\end{oefening}

\begin{oefening}
Beschouw de functie
$$f(x)=\dfrac{1}{2-x}$$
Bepaal grafische de volgende limieten:
\begin{enumerate}[(a)]
  \itemsep.5em
  \item $\displaystyle\lim_{x\to 2} f(x)$
  \item $\displaystyle\lim_{x\underset{<}{\to} 2} f(x)$
  \item $\displaystyle\lim_{x\underset{>}{\to} 2} f(x)$
\end{enumerate}
\end{oefening}

\begin{oefening}
Beschouw de functie
$$f(x)=\dfrac{1}{x^2-4x+4}$$
Bepaal grafische de volgende limieten:
\begin{enumerate}[(a)]
  \itemsep.5em
  \item $\displaystyle\lim_{x\to 2} f(x)$
  \item $\displaystyle\lim_{x\underset{<}{\to} 2} f(x)$
  \item $\displaystyle\lim_{x\underset{>}{\to} 2} f(x)$
\end{enumerate}
\end{oefening}

\subsection{Limieten op oneindig}

Limieten op oneindig karakteriseren een bepaald gedrag van de functiewaarden als de $x$-waarde steeds groter wordt.

\subsubsection*{Eindige limieten op oneindig}

\begin{definitie}
  We zeggen dat een reëel getal $b$ de {\bf limiet is van $f$ voor $x$ gaande naar $+\infty$} als de functiewaarden $f(x)$ willekeurig dicht bij $b$ komen voor willekeurig grote $x$-waarden. We noteren dit als
  $$\lim_{x\to +\infty} f(x)=b$$
\end{definitie}

\begin{oefening}
  Definieer zelf de limiet van een functie $f$ voor $x$ gaande naar $-\infty$.
\end{oefening}

\begin{oefening}
  Teken de grafiek van $f(x)=\dfrac{1}{x}$ en bepaal
  \begin{enumerate}[(a)]
  \itemsep.5em
  \item $\displaystyle \lim_{x\to +\infty} f(x)$
  \item $\displaystyle \lim_{x\to -\infty} f(x)$
  \end{enumerate}
\end{oefening}

\subsubsection*{Oneindige limieten op oneindig}

\begin{definitie}
  We zeggen dat een functie  {\bf $f$ limiet $+\infty$ heeft voor $x$ gaande naar $+\infty$} als de functiewaarden $f(x)$ willekeurig groot worden voor willekeurig grote $x$-waarden. We noteren dit als
  $$\lim_{x\to +\infty} f(x)=+\infty$$
\end{definitie}

\begin{oefening}
  Definieer op analoge wijze
  $$\lim_{x\to +\infty} f(x)=-\infty\;,\qquad\lim_{x\to -\infty} f(x)=+\infty\;,\qquad\lim_{x\to -\infty} f(x)=-\infty$$
\end{oefening}

\begin{oefening}
  Beschouw de functie $f(x)=e^x$ waarvan hieronder de grafiek getekend staat.
  \begin{center}
    \definecolor{cqcqcq}{rgb}{0.75,0.75,0.75}
    \begin{tikzpicture}[line cap=round,line join=round,>=triangle 45,x=1.0cm,y=1.0cm]
      \draw [color=cqcqcq,dash pattern=on 1pt off 1pt, xstep=1.0cm,ystep=1.0cm] (-4.24,-0.69) grid (3.35,4.09);
      \draw[->,color=black] (-4.24,0) -- (3.35,0);
      \foreach \x in {-4,-3,-2,-1,1,2,3}
      \draw[shift={(\x,0)},color=black] (0pt,2pt) -- (0pt,-2pt) node[below] {\footnotesize $\x$};
      \draw[->,color=black] (0,-0.69) -- (0,4.09);
      \foreach \y in {,1,2,3,4}
      \draw[shift={(0,\y)},color=black] (2pt,0pt) -- (-2pt,0pt) node[left] {\footnotesize $\y$};
      \draw[color=black] (0pt,-10pt) node[right] {\footnotesize $0$};
      \clip(-4.24,-0.69) rectangle (3.35,4.09);
      \draw[line width=1.6pt, smooth,samples=100,domain=-4.2:3.3] plot(\x,{exp(\x)});
      \draw (1.24,2.66) node[anchor=north west] {$f(x)=e^x$};
    \end{tikzpicture}
  \end{center}
  Bepaal
  \begin{enumerate}[(a)]
  \itemsep.5em
  \item $\displaystyle \lim_{x\to -\infty} f(x)$
  \item $\displaystyle \lim_{x\to -1} f(x)$
  \item $\displaystyle \lim_{x\to 0} f(x)$
  \item $\displaystyle \lim_{x\to 1} f(x)$
  \item $\displaystyle \lim_{x\to +\infty} f(x)$
  \end{enumerate}
  Welk van de vorige limieten is een oneindige limiet op oneindig?
\end{oefening}

\cleardoublepage
\section{Berekenen van limieten}

\subsection{Rekenregels voor rekenen met oneindig}

Voor oneindige limieten gelden vorige rekenregels enkel als ze welbepaald zijn, dit wil zeggen dat één van de volgende rekenregels toepasbaar is ($k\in\mathbb{R}$):

\begin{multicols}{2}
  \begin{align*}
    (+\infty)+(+\infty) &= +\infty\\
    (-\infty)+(-\infty) &= -\infty\\
    k+(+\infty)&=+\infty\\
    k+(-\infty)&=-\infty\\
    (+\infty)\cdot(+\infty) &= +\infty\\
    (+\infty)\cdot(-\infty) &= -\infty\\
    (-\infty)\cdot(+\infty) &= -\infty\\
    (-\infty)\cdot(-\infty) &= +\infty
  \end{align*}

  \begin{align*}
    k\cdot(+\infty)&=\begin{cases}+\infty & \mbox{ als } k>0\\-\infty & \mbox{ als } k<0\end{cases}\\
    k\cdot(-\infty)&=\begin{cases}-\infty & \mbox{ als } k>0\\+\infty & \mbox{ als } k<0\end{cases}\\
    \dfrac{k}{+\infty}&=0\\
    \dfrac{k}{-\infty}&=0\\
    \dfrac{k}{0} &= +\infty \mbox{ of } -\infty
  \end{align*}
\end{multicols}

De volgende uitdrukkingen hebben geen rekenregels:
$$(\pm\infty)-(\pm\infty)\;,\quad(\pm\infty)+(\mp\infty)$$
$$\dfrac{\pm\infty}{\pm\infty}\;,\quad\dfrac{0}{0}\;,\quad \dfrac{k}{0} $$
$$0\cdot(\pm\infty)\;,\quad 1^{\pm\infty}$$
We noemen ze {\bf onbepaalde vormen (VOV)}, deze kunnen in sommige gevallen
toch berekend worden. We noemen dit de onbepaaldheid opheffen, zie
verder.

\begin{oefening}
Bereken:
\begin{exlist}{2}
  \itemsep.5em
  \item $-\infty\cdot3+10^{100}$
  \item $\dfrac{100}{0}$
  \item $-\dfrac{99}{0}$
  \item $\dfrac{0}{0}$
  \item $(-\infty)^4$
  \item $(-\infty)^5$
  \item $-(-\infty)^3$
\end{exlist}
\end{oefening}

Zoek op internet de naam op van $10^{100}$. Machten kunnen textueel weergegeven worden m.b.v. een \textasciicircum, dus voorgaande getal wordt \texttt{10\textasciicircum 100}.


\needspace{4cm}
\subsection{Basislimieten}

\begin{itemize}
\item $\displaystyle\lim_{x\to a}c=c\;,\qquad\lim_{x\to \pm\infty}c=c\qquad (\mbox{met } c\in\mathbb{R}\mbox{ een constant getal})$
\item $\displaystyle\lim_{x\to a}x=a\;,\qquad\lim_{x\to \pm\infty}x=\pm\infty$
\item $\displaystyle\lim_{x\to+\infty}x^k = +\infty\qquad(k\in\mathbb{N}_0)\;,\qquad \displaystyle\lim_{x\to-\infty}x^k = \begin{cases}+\infty &\mbox{ als $k$ even is}\\-\infty &\mbox{ als $k$ oneven is} \end{cases}$
\item $\displaystyle\lim_{x\to\pm\infty}\dfrac{1}{x}=0\;,\qquad\lim_{x\underset{<}{\to}0}\dfrac{1}{x}=-\infty\;,\qquad \displaystyle\lim_{x\underset{>}{\to}0}\dfrac{1}{x}=+\infty$
%\item $\displaystyle\lim_{x\to\pm\infty}\left(a_nx^n+a_{n-1}x^{n-1}+\cdots+a_1x+a_0\right) = \lim_{x\to\pm\infty}a_nx^n$
\end{itemize}

\begin{oefening}
  Bereken volgende basislimieten:
  \begin{exlist}{2}
  \item $\displaystyle \lim_{x\to 3}42$
  \item $\displaystyle \lim_{x\to -\infty}24$
  \item $\displaystyle \lim_{x\to 3}x$
  \item $\displaystyle \lim_{x\to -\infty}x$
  \item $\displaystyle \lim_{x\to -\infty}x^3$
  \item $\displaystyle \lim_{x\to -\infty}x^6$
  \item $\displaystyle \lim_{x\to 0}\dfrac{1}{x}$
%  \item $\displaystyle \lim_{x\to -\infty}(2x^3-3x^2+4x-6)$
%  \item $\displaystyle \lim_{x\to -\infty}(7x^4-8x^2-9)$
  \end{exlist}
\end{oefening}

\subsection{Rekenregels voor limieten}

Voor eindige limieten hebben we de volgende rekenregels:
\begin{align*}
  \displaystyle\lim_{x\to a}f(x)=b\quad\mbox{en}\quad \lim_{x\to a}g(x)=c &\Rightarrow \lim_{x\to a}\left(f(x)\pm g(x)\right)=b\pm c\\
  \displaystyle\lim_{x\to a}f(x)=b\quad\mbox{en}\quad k\in\mathbb{R} &\Rightarrow \lim_{x\to a} kf(x)=kb\\
  \displaystyle\lim_{x\to a}f(x)=b\quad\mbox{en}\quad \lim_{x\to a}g(x)=c &\Rightarrow \lim_{x\to a}\left(f(x)\cdot g(x)\right)=b\cdot c\\
  \displaystyle\lim_{x\to a}f(x)=b\quad\mbox{en}\quad \lim_{x\to a}g(x)=c \neq 0 &\Rightarrow \lim_{x\to a}\dfrac{f(x)}{g(x)}=\dfrac{b}{c}\\
  \displaystyle\lim_{x\to a}f(x)=b\quad\mbox{en}\quad n\in\mathbb{N}_0 &\Rightarrow \lim_{x\to a} [f(x)]^n=b^n\\
\end{align*}

\begin{oefening}
  Bereken volgende limieten
  \begin{exlist}{2}
  \item $\displaystyle \lim_{x\to 3}x^2+x$
  \item $\displaystyle \lim_{x\to 4}3x^2$
  \end{exlist}
\end{oefening}

\begin{oefening}
  Bereken volgende limieten op twee manieren
  \begin{exlist}{2}
  \item $\displaystyle \lim_{x\to 3}x^2$ en $\displaystyle \lim_{x\to 3}\left(x\cdot x\right)$
  \item $\displaystyle \lim_{x\to 4}x^3$ en $\displaystyle \left(\lim_{x\to 4}x\right)^3$
  \end{exlist}
\end{oefening}

\needspace{4cm}
\subsection{Limiet van een veeltermfunctie}

We kunnen twee gevallen onderscheiden.

\subsubsection*{Geval 1: limiet naar een reëel getal}

De limiet van een veeltermfunctie in een reëel getal is zeer eenvoudig, daar valt de limiet in $a$ samen met de functiewaarde in $a$. We vinden dus

\begin{eigenschap}
  $$\lim_{x\to a}f(x)=f(a) \qquad \mbox{ met $f(x)$ een veeltermfunctie en $a\in\mathbb{R}$}$$
\end{eigenschap}

\begin{oefening}
  Bereken
  \begin{enumerate}[(a)]
  \itemsep.5em
  \item $\displaystyle\lim_{x\to1}\left(5x^4+2\right)$
  \item $\displaystyle\lim_{x\to-2}\left(3x^3+2x^2-x+3\right)$
  \item $\displaystyle\lim_{x\to0}\left(89x^5-65x^2-100x+1\right)$
  \end{enumerate}
\end{oefening}

\subsubsection*{Geval 2: limiet naar oneindig}

Voor de limiet op oneindig van een veeltermfunctie kan gebruik worden gemaakt van:

\begin{eigenschap}
  $$\lim_{x\to\pm\infty}\left(a_nx^n+a_{n-1}x^{n-1}+\cdots+a_1x+a_0\right) = \lim_{x\to\pm\infty}a_nx^n$$
\end{eigenschap}

\begin{oefening}
  Geef de vorige eigenschap met woorden. Volgende woorden moeten hierin zeker voorkomen:\\
  {\em veeltermfunctie, limiet, oneindig, graad, term}.
\end{oefening}

Dat deze definitie klopt kunnen we bijvoorbeeld zien m.b.v. volgende voorbeeld.

\paragraph*{Voorbeeld:}

Beschouw de functie
\[f(x)=-0.0001x^6 - 0.001x^4 + 0.05x^3 + x^2 - 2x\]

Tekenen we de grafiek van deze functie in het interval $[-5,5]$

\begin{center}
  \begin{tikzpicture}[line cap=round,line join=round,>=triangle 45,x=1.0cm,y=1.0cm]
    \begin{axis}[
      x=1.0cm,y=0.3cm,
      axis lines=middle,
      xmin=-5.0,
      xmax=5.0,
      ymin=-2.0,
      ymax=15.0,
      xtick={-5.0,0.0,...,5.0},
      ytick={-0.0,5.0,...,15.0},]
      \clip(-5.,-2.) rectangle (5.,15.);
      \draw[line width=2.pt,smooth,samples=100,domain=-5.0:5.0] plot(\x,{0-1.0E-4*(\x)^(6.0)-0.001*(\x)^(4.0)+0.05*(\x)^(3.0)+(\x)^(2.0)-2.0*(\x)});
      \begin{scriptsize}
        \draw (-8.706835340066094,-21.52772092098425) node {$f$};
      \end{scriptsize}
    \end{axis}
  \end{tikzpicture}
\end{center}

Dan lijkt het in eerste instantie dat we een dalparabool krijgen, dit omdat de coëfficiënten bij de termen met graad hoger dan twee zo klein zijn. Bekijken we nu echter dezelfde functie in het interval $[-10,10]$ dan krijgen we een gans ander beeld. Geleidelijk aan krijgt de hoogste term $-0.0001x^6$ de overhand.

\begin{center}
  \begin{tikzpicture}[line cap=round,line join=round,>=triangle 45,x=1.0cm,y=1.0cm]
    \begin{axis}[
      x=0.5cm,y=0.2cm,
      axis lines=middle,
      xmin=-10.0,
      xmax=12.0,
      ymin=-20.0,
      ymax=45.0,
      xtick={-10.0,-5.0,...,10.0},
      ytick={-20.0,-15.0,...,45.0},]
      \clip(-10.,-20.) rectangle (12.,45.);
      \draw[line width=2.pt,smooth,samples=100,domain=-10.0:12.0] plot(\x,{0-1.0E-4*(\x)^(6.0)-0.001*(\x)^(4.0)+0.05*(\x)^(3.0)+(\x)^(2.0)-2.0*(\x)});
      \begin{scriptsize}
        \draw (-8.706835340066094,-21.52772092098425) node {$f$};
      \end{scriptsize}
    \end{axis}
  \end{tikzpicture}
\end{center}

\begin{oefening}
  Bereken
  \begin{enumerate}[(a)]
  \itemsep.5em
  \item $\displaystyle\lim_{x\to-\infty}\left(10x^4+x^3\right)$
  \item $\displaystyle\lim_{x\to-\infty}\left(52x^4+x^5+34\right)$
  \item $\displaystyle\lim_{x\to+\infty}\left(2x^3+x^2+1\right)$
  \end{enumerate}
\end{oefening}

\subsection{Limiet van een rationale functie}

Er zijn vier gevallen voor het berekenen van de limiet van een rationale functie $\displaystyle\lim_{x\to a}\dfrac{f(x)}{g(x)}$.

\subsubsection*{Geval 1: $a$ behoort tot het domein}

Heel eenvoudig, in dit geval valt de limiet voor $x$ naar $a$ samen met de functiewaarde in $a$. We krijgen dus volgende eigenschap:

\begin{eigenschap}
  $$\lim_{x\to a}\dfrac{f(x)}{g(x)}=\dfrac{f(a)}{g(a)} \qquad \mbox{als $g(a)\neq 0$}$$
\end{eigenschap}

\begin{oefening}
  Bereken
  \begin{enumerate}[(a)]
  \itemsep.5em
  \item $\displaystyle\lim_{x\to2}\dfrac{x^2-1}{x^2-4x+3}$
  \item $\displaystyle\lim_{x\to-3}\dfrac{x^2-9}{x+5}$
  \end{enumerate}
\end{oefening}

\subsubsection*{Geval 2: $a$ is een nulwaarde van de noemer en geen nulwaarde van de teller}

In dit geval krijgen we na uitrekenen van de limiet dat deze gelijk is aan $\dfrac{k}{0}$, met $k\neq 0$. Bij limieten is deze uitdrukking wel degelijk bepaald. Dit kan gelijk zijn aan $+\infty$ of $-\infty$, een tekenonderzoek moet uitwijzen aan wat.

\begin{eigenschap}
  $$\lim_{x\to a}\dfrac{f(x)}{g(x)}=\dfrac{k}{0}=\underbrace{+\infty \mbox{ of } -\infty}_{\mbox{tekenonderzoek}} \qquad \mbox{als $f(a)\neq0$ en $g(a)=0$}$$
\end{eigenschap}

\begin{oefening}
  Bereken
  \begin{enumerate}[(a)]
  \itemsep.5em
  \item $\displaystyle\lim_{x\to3}\dfrac{x^2-1}{(x-3)^2}$
  \item $\displaystyle\lim_{x\underset{>}{\to}-3}\dfrac{6x^2-x-1}{x+3}$
  \end{enumerate}
\end{oefening}

\subsubsection*{Geval 3: $a$ is een nulwaarde van de teller en van de noemer}

Hier zullen we de onbepaaldheid $\dfrac{0}{0}$ krijgen die we kunnen opheffen. Hiervoor moet teller en noemer ontbonden worden in factoren, bijvoorbeeld door gebruik te maken van Horner. In de limiet mogen we in teller en noemer gelijke factoren dan wegdelen en krijgen we zo een nieuwe limiet die hopelijk dan welbepaald is. Het kan echter wel gebeuren dat we de methode verschillende keren moeten herhalen.

\begin{eigenschap}
  $$\lim_{x\to a}\dfrac{f(x)}{g(x)}=\dfrac{"0"}{0}=\lim_{x\to a}\dfrac{(x-a)\;\;f_r(x)}{(x-a)\underbrace{g_r(x)}_{\mbox{Horner}}}=\lim_{x\to a}\dfrac{f_r(x)}{g_r(x)}$$
\end{eigenschap}

\begin{oefening}
  Bereken
  \begin{enumerate}[(a)]
  \itemsep.5em
  \item $\displaystyle\lim_{x\to1}\dfrac{x^2-1}{x^2-4x+3}$
  \item $\displaystyle\lim_{x\to2}\dfrac{5x^2-20x+20}{x^3-4x^2+4x}$
  \end{enumerate}
\end{oefening}

\subsubsection*{Geval 4: limiet naar oneindig van een rationale functie}

Deze limiet zal gelijk zijn aan de limiet van het quotiënt van de termen met de hoogste graad in teller en noemer, in symbolen:

\begin{eigenschap}
  $$\lim_{x\to \pm\infty}\dfrac{\left(a_nx^n+a_{n-1}x^{n-1}+\cdots+a_1x+a_0\right)}{\left(b_mx^m+b_{m-1}x^{m-1}+\cdots+b_1x+b_0\right)}=\lim_{x\to \pm\infty}\dfrac{a_nx^n}{b_mx^m}$$
\end{eigenschap}

Uitwerken van het rechterlid kan na vereenvoudigen.

\begin{oefening}*
  Bewijs dat
    $$\lim_{x\to \pm\infty}\dfrac{\left(a_nx^n+a_{n-1}x^{n-1}+\cdots+a_1x+a_0\right)}{\left(b_mx^m+b_{m-1}x^{m-1}+\cdots+b_1x+b_0\right)}=\lim_{x\to \pm\infty}\dfrac{a_nx^n}{b_mx^m}$$

\end{oefening}

\begin{oefening}
  Bereken
  \begin{enumerate}[(a)]
  \itemsep.5em
  \item $\displaystyle\lim_{x\to+\infty}\dfrac{5x^2+3x-7}{2x^2-x+4}$
  \item $\displaystyle\lim_{x\to2}\dfrac{3x-4}{x^2+3x+4}$
  \item $\displaystyle\lim_{x\to+\infty}\dfrac{3x-4}{x^2+3x+4}$
  \end{enumerate}
\end{oefening}

\needspace{3cm}
\subsection{Extra oefeningen op limieten}

\begin{oefening}
  Bepaal de volgende limieten
  \begin{multicols}{3}
  \begin{enumerate}[(a)]
  \itemsep1em
  \item $\displaystyle\lim_{x\to 2}\left(x^3+2x-1\right)$
  \item $\displaystyle\lim_{x\to +\infty}\left(x^3+2x-1\right)$
  \item $\displaystyle\lim_{x\to -\infty}\left(x^3+2x-1\right)$
  \item $\displaystyle\lim_{x\to 3}\dfrac{3-x}{x^2-9}$
  \item $\displaystyle\lim_{x\to -2}\dfrac{x+2}{x^2-4}$
  \item $\displaystyle\lim_{x\underset{>}{\to} 2}\dfrac{x+2}{x^3-8}$
  \item $\displaystyle\lim_{x\to 1}\dfrac{x^2+2x-3}{x^3+3x^2-4}$
  \item $\displaystyle\lim_{x\to +\infty}\dfrac{x^2+2x-3}{x^3+3x^2-4}$
  \item $\displaystyle\lim_{x\to -\infty}\dfrac{x^2+2x-3}{x^3+3x^2-4}$
  \item $\displaystyle\lim_{x\to +\infty}\dfrac{x^3+2x-3}{x^3+3x^2-4}$
  \item $\displaystyle\lim_{x\to +\infty}\dfrac{x^4+2x-3}{x^3+3x^2-4}$
  \item $\displaystyle\lim_{x\to -\infty}\dfrac{x^2(x^2-1)^2}{(2x^2+4)^3}$
  \end{enumerate}
  \end{multicols}
\end{oefening}

\begin{oefening}
  Bepaal de volgende limieten
  \begin{multicols}{2}
  \begin{enumerate}[(a)]
  \itemsep1em
  \item $\displaystyle\lim_{x\to 2}\dfrac{x^2-4}{x-2}$
  \item $\displaystyle\lim_{x\to 2}\dfrac{x^2+3x-10}{x-2}$
  \item $\displaystyle\lim_{x\to 2}\dfrac{2x^2-4x}{x^2-5x+6}$
  \item $\displaystyle\lim_{x\to 2}\dfrac{x^4-16}{x-2}$
  \end{enumerate}
  \end{multicols}
\end{oefening}

\begin{oefening}
  Bepaal de volgende limieten
  \begin{multicols}{3}
  \begin{enumerate}[(a)]
  \itemsep1em
  \item $\displaystyle\lim_{x\to +\infty}\dfrac{4x^2+16}{2x^2-4}$
  \item $\displaystyle\lim_{x\to +\infty}\dfrac{4x^2+16}{2x-4}$
  \item $\displaystyle\lim_{x\to +\infty}\dfrac{4x+16}{2x^2-4}$
  \end{enumerate}
  \end{multicols}
\end{oefening}

\begin{oefening}
  Bepaal de volgende limieten
  \begin{multicols}{2}
  \begin{enumerate}[(a)]
  \itemsep1em
  \item $\displaystyle\lim_{t\to 4}\dfrac{t^4}{t^2-4}$
  \item $\displaystyle\lim_{x\to 0}\dfrac{\frac{1}{x-8}-\frac{1}{8}}{x}$
  \item $\displaystyle\lim_{x\to +\infty}\left(-2x^4+3x^3-4x^2+5x-6\right)$
  \item $\displaystyle\lim_{z\to 2}\dfrac{z^3-8}{z-2}$
  \item $\displaystyle\lim_{x\to -\infty}\dfrac{5x^2}{x+3}$
  \item $\displaystyle\lim_{x\to -1}\dfrac{x^3+2x^2-x-2}{x^3+4x^2-x-4}$
  \item $\displaystyle\lim_{t\to 2}\dfrac{t^5-32}{t-2}$
  \item $\displaystyle\lim_{h\to 0}\dfrac{2(-3+h)^2-18}{h}$
  \item $\displaystyle\lim_{x\to -3}\dfrac{x^2-2x-3}{x^2+6x+9}$
  \item $\displaystyle\lim_{x\to -1}\left(-2x^4+3x^3-4x^2+5x-6\right)$
  \end{enumerate}
  \end{multicols}
\end{oefening}

\begin{oefening}{\em\small Bron: Examenvraag 1ETEW -- K.U.Leuven}\\
Wanneer men in de context van limieten van functies $f,g:\mathbb{R}\to\mathbb{R}$ zegt dat $0\times(+\infty)$ een
{\em onbepaalde vorm} is, dan bedoelt men
  \begin{enumerate}[(A)]
  \itemsep1em
  \item als $\lim_{x\to a}f(x)=0$ en $\lim_{x\to a}g(x)=+\infty$, dan bestaat $\lim_{x\to a}f(x)\cdot g(x)$ niet
  \item als $\lim_{x\to a}f(x)=0$ en $\lim_{x\to a}g(x)=+\infty$, dan bestaat $\lim_{x\to a}f(x)\cdot g(x)$ eventueel wel, maar hij kan hoe dan ook niet berekend worden
  \item als $\lim_{x\to a}f(x)=0$ en $\lim_{x\to a}g(x)=+\infty$, dan kan in het algemeen niets besloten worden over het bestaan en de eventuele waarde van $\lim_{x\to a}f(x)\cdot g(x)$
  \item als $\lim_{x\to a}f(x)=0$ en $\lim_{x\to a}g(x)=+\infty$, dan bestaat $\lim_{x\to a}f(x)\cdot g(x)$ zeker, maar de waarde van deze limiet hangt af van wat $f$ en $g$ precies zijn
  \end{enumerate}
\end{oefening}

\begin{oefening}{\em\small Bron: Oefenvraag 1ETEW -- K.U.Leuven/KULAK}\\
$$\lim_{x\to+\infty}\dfrac{x^{10}+(x-1)^{10}+(x-2)^{10}+\cdots+(x-10)^{10}}{x^{10}-10^{10}}=$$
  \begin{enumerate}[(A)]
  \itemsep1em
  \item 0
  \item 1
  \item 10
  \item 11
  \end{enumerate}
\end{oefening}

\begin{oefening}{\em\small Bron: 1 HW-VLEKHO, Wiskunde voor bedrijfseconomen}\\
$$\lim_{x\to3}\left(\dfrac{2}{\left(x-3\right)^4}-\dfrac{5}{\left(x-3\right)^3}\right)=$$
  \begin{enumerate}[(A)]
  \itemsep1em
  \item 0
  \item $+\infty$
  \item bestaat niet
  \item $-\infty$
  \end{enumerate}
\end{oefening}

\cleardoublepage
\section{Asymptoten}

Een {\bf asymptoot} van een functie $f$ is een rechte waarnaar de grafiek van die functie $f$ steeds dichter nadert. Daar rechten, verticaal, horizontaal of schuin kunnen zijn maken we volgende opdeling.

\subsection{Verticale asymptoot}

\subsubsection*{Voorbeeld}

\begin{minipage}{0.5\textwidth}
Beschouw volgende schets:
\begin{center}
\begin{tikzpicture}[scale=0.7,line cap=round,line join=round,>=triangle 45,x=1.0cm,y=1.0cm]
\draw[->,color=black] (-4.3,0) -- (6.5,0);
\draw[->,color=black] (0,-3.3) -- (0,6.44);
\clip(-4.3,-3.3) rectangle (6.5,6.44);
\draw[line width=2.4pt, smooth,samples=100,domain=-4.3:1.9] plot(\x,{(3*(\x)-5)/((\x)-2)});
\draw[line width=2.4pt, smooth,samples=100,domain=2.1:6.5] plot(\x,{(3*(\x)-5)/((\x)-2)});
\draw [line width=1.2pt,dash pattern=on 2pt off 2pt] (2,-3.3) -- (2,6.44);
\draw (2.24,-0.3) node[anchor=north west] {a};
\draw (3.66,4.3) node[anchor=north west] {f};
\end{tikzpicture}
\end{center}
\end{minipage}
\begin{minipage}{0.5\textwidth}
We zien dat we een verticale asymptoot hebben:
$$\mbox{V.A.: }x=a$$

We bepalen m.b.v. limieten:
$$\lim_{x\underset{<}{\to}a} f(x)=-\infty$$
$$\lim_{x\underset{>}{\to}a} f(x)=+\infty$$
\vfill
\end{minipage}

We krijgen:

\begin{definitie}
Als $f$ minstens een interval $]a-\delta,a[$ of $]a,a+\delta[$ omvat voor een willekeurige kleine $\delta>0$ en als $a\notin \dom f$ dan is $x=a$ een {\bf verticale asymptoot} van $f$ in $a$ als één van volgende limieten waar is:
\begin{multicols}{3}
  \begin{itemize}
    \item $\lim_{x\to a} f(x)=+\infty$
    \item $\lim_{x\to a} f(x)=-\infty$
    \item $\lim_{x\underset{<}{\to}a} f(x)=-\infty$
    \item $\lim_{x\underset{<}{\to}a} f(x)=+\infty$
    \item $\lim_{x\underset{>}{\to}a} f(x)=-\infty$
    \item $\lim_{x\underset{>}{\to}a} f(x)=+\infty$
  \end{itemize}
\end{multicols}
\end{definitie}

\subsubsection*{Opsporen van een verticale asymptoot}

Er geldt dat een functie $f$ een verticale asymptoot heeft in elk nulpunt van de noemer die een hogere multipliciteit heeft dan een zelfde nulpunt van de teller. Men kan echter eenvoudig de limiet uitrekenen in elke nulwaarde van de noemer. Als de limiet $\pm \infty$ is, dan weet men dat er daar een verticale asymptoot is.

\subsubsection*{Voorbeeld}
$$f(x)=\dfrac{2x-4}{x-1}$$

\subsubsection*{Tegenvoorbeeld}
$$f(x)=\dfrac{2x-4}{x-2}$$

\begin{oefening}
Bepaal alle verticale asymptoten:
\begin{enumerate}[(a)]
  \itemsep.5em
  \item $f(x)=\dfrac{1}{x^2}$
  \item $f(x)=\dfrac{1}{x^2-2x+1}$
  \item $f(x)=\dfrac{2x-6}{x^2-2x-3}$
\end{enumerate}
\end{oefening}

\subsection{Horizontale asymptoot}

\subsubsection*{Voorbeeld}

\begin{minipage}{0.5\textwidth}
Beschouw volgende schets:
\begin{center}
\begin{tikzpicture}[scale=0.7,line cap=round,line join=round,>=triangle 45,x=1.0cm,y=1.0cm]
\draw[->,color=black] (-4.3,0) -- (6.5,0);
\draw[->,color=black] (0,-3.3) -- (0,6.44);
\clip(-4.3,-3.3) rectangle (6.5,6.44);
\draw[line width=2.4pt, smooth,samples=100,domain=-4.3:1.9] plot(\x,{(3*(\x)-5)/((\x)-2)});
\draw[line width=2.4pt, smooth,samples=100,domain=2.1:6.5] plot(\x,{(3*(\x)-5)/((\x)-2)});
\draw [line width=1.2pt,dash pattern=on 2pt off 2pt] (-4.3,3) -- (6.5,3);
\draw (-0.2,3.3) node[anchor=south east] {b};
\draw (3.66,4.3) node[anchor=north west] {f};
\end{tikzpicture}
\end{center}
\end{minipage}
\begin{minipage}{0.5\textwidth}
We zien dat we een horizontale asymptoot hebben:
$$\mbox{H.A.: }y=b$$

We bepalen m.b.v. limieten:
$$\lim_{x\to -\infty} f(x)=b$$
$$\lim_{x\to +\infty} f(x)=b$$
\vfill
\end{minipage}

\needspace{2cm}
\begin{definitie}
Een horizontale rechte met vergelijking $y=b$ is een {\bf horizontale asymptoot} van $f$ als één van de volgende limieten waar is:
\begin{multicols}{2}
\begin{itemize}
  \item $\lim_{x\to+\infty} f(x)=b$
  \item $\lim_{x\to-\infty} f(x)=b$
\end{itemize}
\end{multicols}
\end{definitie}

\subsubsection*{Opsporen van een horizontale asymptoot}

Er geldt dat een rationale functie $f$ enkel horizontale asymptoten heeft als de graad van teller kleiner of gelijk is aan de graad van de noemer. In de praktijk kunnen we eenvoudig de limiet van de functie $f$ uitrekenen in $-\infty$ en in $+\infty$. Als één van de limieten gelijk is aan een getal $b\in\mathbb{R}$, dus verschillend van $\pm\infty$ dan is $y=b$ een horizontale asymptoot.

\subsubsection*{Voorbeeld}
$$f(x)=\dfrac{2x-4}{x-1}$$

\subsubsection*{Tegenvoorbeeld}
$$f(x)=\dfrac{x^2-4}{x+2}$$

\begin{oefening}
Bepaal alle horizontale asymptoten:
\begin{enumerate}[(a)]
  \itemsep.5em
  \item $f(x)=\dfrac{1}{x^2}$
  \item $f(x)=\dfrac{3x^2+2x-1}{x^2-2x+4}$
  \item $f(x)=\dfrac{2x-6}{x^2-2x-3}$
  \item $f(x)=x^2$
\end{enumerate}
\end{oefening}

\subsection{Schuine asymptoot}

\subsubsection*{Voorbeeld}

\begin{minipage}{0.5\textwidth}
Beschouw volgende schets:
\begin{center}
\begin{tikzpicture}[scale=0.7,line cap=round,line join=round,>=triangle 45,x=1.0cm,y=1.0cm]
\draw[->,color=black] (-4.3,0) -- (6.5,0);
\foreach \x in {-4,-2,2,4,6}
\draw[shift={(\x,0)},color=black] (0pt,2pt) -- (0pt,-2pt) node[below] {\footnotesize $\x$};
\draw[->,color=black] (0,-3.3) -- (0,6.44);
\foreach \y in {-2,2,4,6}
\draw[shift={(0,\y)},color=black] (2pt,0pt) -- (-2pt,0pt) node[left] {\footnotesize $\y$};
\draw[color=black] (0pt,-10pt) node[right] {\footnotesize $0$};
\clip(-4.3,-3.3) rectangle (6.5,6.44);
\draw[line width=2.4pt, smooth,samples=100,domain=-4.3:1.9] plot(\x,{((\x)^2-2)/(2*(\x)-4)});
\draw[line width=2.4pt, smooth,samples=100,domain=2.1:6.5] plot(\x,{((\x)^2-2)/(2*(\x)-4)});
\draw (1.8,1.9) node[rotate=26,anchor=north west] {$y=mx+q$};
\draw (3.66,4.4) node[anchor=north west] {$f$};
\draw [line width=1.2pt,dash pattern=on 2pt off 2pt,domain=-4.3:6.5] plot(\x,{(--1--0.5*\x)/1});
\end{tikzpicture}
\end{center}
\end{minipage}
\begin{minipage}{0.5\textwidth}
We zien dat we een schuine asymptoot hebben:
$$\mbox{S.A.: }y=mx+q$$
\vfill
\end{minipage}

\paragraph*{Opmerking:} Om inzicht te krijgen in bovenstaande figuur teken je volgende functies, telkens in een ander kleur:
\begin{itemize}
  \item $f(x)=\dfrac{x^2-2}{2x-4}$
  \item $g(x)=\dfrac{1}{2}x+1$
  \item $h(x)=f(x)-g(x)$
\end{itemize}

\begin{definitie}
Een schuine rechte met vergelijking $y=mx+q$ is een {\bf schuine asymptoot} van $f$ als één van de volgende limieten waar is:
\begin{multicols}{2}
\begin{itemize}
  \item $\lim_{x\to+\infty} \left(f(x)-(mx+q)\right)=0$
  \item $\lim_{x\to-\infty} \left(f(x)-(mx+q)\right)=0$
\end{itemize}
\end{multicols}
\end{definitie}

\subsubsection*{Opsporen van een horizontale asymptoot}

Er geldt dat een rationale functie enkel schuine asymptoten heeft als de graad van de teller precies één meer is dan de graad van de noemer. We vinden
\begin{align*}
  m&=\lim_{x\to\pm\infty}\dfrac{f(x)}{x}\\
  q&=\lim_{x\to\pm\infty}\left(f(x)-mx\right)
\end{align*}

\subsubsection*{Voorbeeld}
$$f(x)=\dfrac{x^2-x-1}{x+1}$$

\paragraph*{Opmerking:} Een horizontale asymptoot is een speciaal geval van de schuine asymptoot, maar dan met $m=0$. Het spreekt voor zich dat een functie geen horizontale en schuine asymptoot naar $+\infty$ of naar $-\infty$ kan hebben.

\begin{oefening}
Bepaal alle schuine asymptoten:
\begin{enumerate}[(a)]
  \itemsep.5em
  \item $f(x)=\dfrac{x^2-2}{2x-4}$
  \item $f(x)=\dfrac{3x^2+2x-1}{x^2-2x+4}$
\end{enumerate}
\end{oefening}

\needspace{3cm}
\subsection{Extra oefeningen op asymptoten}

\begin{oefening} % http://www.purplemath.com/modules/asymtote4.htm
Bepaal alle asymptoten\\
\begin{multicols}{2}
\begin{enumerate}[(a)]
  \itemsep1em
  \item $\displaystyle y=\dfrac{2x^2}{x+1}$
  \item $\displaystyle y=\dfrac{x^2+3x+1}{4x^2-9}$
  \item $\displaystyle y=\dfrac{x+3}{x^2+9}$
  \item $\displaystyle y=\dfrac{x^2-x-2}{x-2}$
  \item $\displaystyle y=\dfrac{-3x^2+2}{x-1}$
  \item $\displaystyle y=\dfrac{x^2+3x+2}{x-2}$
  \item $\displaystyle y=\dfrac{2x^3+4x^2-9}{3-x^2}$
\end{enumerate}
\end{multicols}
\end{oefening}

\begin{oefening} % http://www.math-exercises.com/analysis-of-functions/asymptotes-of-a-function
Bepaal alle asymptoten\\
\begin{multicols}{3}
\begin{enumerate}[(a)]
  \itemsep1em
  \item $\displaystyle f(x)=\dfrac{x}{x+4}$
  \item $\displaystyle f(x)=\dfrac{1-x^2}{x-2}$
  \item $\displaystyle f(x)=\dfrac{2x^2}{2x-1}$
  \item $\displaystyle f(x)=\dfrac{x^2+1}{x}$
  \item $\displaystyle f(x)=\dfrac{2x^2-1+3x^3}{3-2x^2}$
  \item $\displaystyle f(x)=\dfrac{x^2-9}{1-x}$
  \item $\displaystyle f(x)=\dfrac{x^2+6x}{x+2}$
  \item $\displaystyle f(x)=\dfrac{2x}{x^2+1}$
  \item $\displaystyle f(x)=3x+\dfrac{3}{x-2}$
  \item $\displaystyle f(x)=\dfrac{2x^3-5x^2+x-4}{4-x^2}$
  \item $\displaystyle f(x)=\dfrac{x^2-3}{x^3-1}$
  \item $\displaystyle f(x)=\dfrac{x^3-3}{x^3-1}$
  \item $\displaystyle f(x)=\dfrac{2x^2-5x}{x^2+1}$
  \item $\displaystyle f(x)=\dfrac{2x}{1-3x}$
  \item $\displaystyle f(x)=\dfrac{3x^2-2x^3+4}{4-4x+x^2}$
  \item $\displaystyle f(x)=\dfrac{x-1}{x^3-1}$
  \item $\displaystyle f(x)=\dfrac{2x^2+x+1}{8x}$
  \item $\displaystyle f(x)=\dfrac{1+x-3x^3}{x^2+x-2}$
\end{enumerate}
\end{multicols}
\end{oefening}

\end{document}


