\documentclass[a4paper,12pt]{article}

\input{../gowiz.tex}

\usepackage{versions}
%\includeversion{theorie}
\excludeversion{theorie}

\begin{document}

\pagestyle{fancy}
\lhead{}
\rhead{Oefeningen Integraalrekening}

\begin{theorie}

\thispagestyle{empty}
\begin{center}
  \begin{mdframed}
  \centering
  \fontsize{40}{50}\selectfont Integraalrekening
  \end{mdframed}
  \vfill
  \vfill
\end{center}
\subsection*{Doelstelling}
Je \hfill  {\scriptsize(LP 2006-059, LI 1.10)}
\begin{itemize}
\end{itemize}


\pagestyle{empty}
\mbox{}
\newpage
\clearpage
\thispagestyle{empty}
%\mbox{}
{\small \tableofcontents}
\newpage
\clearpage
\pagenumbering{arabic}

\pagestyle{fancy}
\lhead{}
\rhead{Goniometrische formules}

\end{theorie}

\onehalfspacing

\section{Differentiaal van een functie}


\begin{oefening}
Bepaal de differentiaal van de volgende functies
\begin{enumerate}[(a)]
\itemsep.5em
  \item $\displaystyle 5+3x^2+8x^4$
  \item $\displaystyle (x^2+x^4+x^6)^4$
  \item $\displaystyle \sqrt{x^2+2x}$
  \item $\displaystyle \dfrac{x^3+3x^2+3x+1}{x+1}$
  \item $\displaystyle e^{\sqrt[3]{3x}}$
  \item $\displaystyle \dfrac{1}{\ln 10}\cdot 10^{x^2}$
  \item $x\sin x + \cos x$
\end{enumerate}
\end{oefening}

\begin{oefening}
Benader
\begin{enumerate}[(a)]
\itemsep.5em
  \item $\displaystyle \sqrt[3]{9}$
  \item $\displaystyle \sqrt{24}$
  \item $\displaystyle e^{0.2}$
  \item $\displaystyle \sqrt{9.5}$
\end{enumerate}
\end{oefening}

\begin{oefening} % bron http://info.math4all.nl/MathAdore/vb-bb43-ex2b.html
De luchtdruk $p$ (in hectopascal $\hPa$) hangt af van de hoogte $h$ in $\km$ boven het aardoppervlak. In een luchtballon is de luchtdruk gemakkelijk te meten en wordt daaruit de hoogte berekend met de formule:
$$h = -6.5 \log \frac{p}{p_0}$$
Hierin is $p_0$ de luchtdruk op zeeniveau. Neem aan dat $p_0 = 1000 \hPa$.
Bereken nu de hoogte en de snelheid waarmee $h(p)$ verandert als $p = 920 \hPa$ wordt gemeten. 
\end{oefening}

\pagebreak
\section{Bepaalde integraal}

\begin{oefening}
Bereken volgende bepaalde integralen
\begin{multicols}{2}
\begin{enumerate}[(a)]
\itemsep1em
  \item $\displaystyle \int_0^4 x \;dx$
  \item $\displaystyle \int_1^3 x^4 \;dx$
  \item $\displaystyle \int_{-4}^4 \;dx$
  \item $\displaystyle \int_{-1}^2 (3x^2+2x-1)\;dx$
  \item $\displaystyle \int_{-4}^3 (6x^8+12x^5-\dfrac{1}{2}x^2)\;dx$
  \item $\displaystyle \int_{0}^5 (3x+5)^2\;dx$
  \item $\displaystyle \int_1^2 \dfrac{1}{x^2}\;dx$
  \item $\displaystyle \int_{-3}^{-1} \dfrac{4}{3x^3} \;dx$
  \item $\displaystyle \int_4^{16} \sqrt{x} \;dx$
  \item $\displaystyle \int_8^{27} \sqrt[3]{x} \;dx$
  \item $\displaystyle \int_2^8 \sqrt{2x} \;dx$
  \item $\displaystyle \int_8^{27} \dfrac{1}{\sqrt[3]{x}} \;dx$
  \item $\displaystyle \int_3^{16/3} \dfrac{x}{\sqrt{3x}} \;dx$
  \item $\displaystyle \int_{0}^3 (x+1)^3\;dx$
  \item $\displaystyle \int_{0}^2 (x+2)^6\;dx$
  \item $\displaystyle \int_{-1}^3 \dfrac{x+1}{x}\;dx$
  \item $\displaystyle \int_{-4}^4 \dfrac{x^2+2x+2}{4x^2}\;dx$
  \item $\displaystyle \int_{4}^0 \sqrt{x}(x-2)\;dx$
\end{enumerate}
\end{multicols}
\end{oefening}

\begin{oefening}
Bereken
\begin{multicols}{2}
\begin{enumerate}[(a)]
\itemsep1em
  \item $\displaystyle \int_1^2 y^2+y^{-2} \;dy$
  \item $\displaystyle \int_{-1}^2 y^2+y^{-2} \;dy$
  \item $\displaystyle \int_1^2 \dfrac{2\omega^5 - \omega + 3}{\omega^2} \;d\omega$
  \item $\displaystyle \int_{25}^{-10} \;dR$
\end{enumerate}
\end{multicols}
\end{oefening}

\begin{oefening} % source http://tutorial.math.lamar.edu/Classes/CalcI/ComputingDefiniteIntegrals.aspx
Gegeven
$$f(x)=\begin{cases}6 & \mbox{ if } x > 1\\3x^2 & \mbox{ if } x \leq 1\\\end{cases}$$
Bepaal
$$\displaystyle \int_{10}^{22} f(x) \;dx \qquad\mbox{ en }\qquad \displaystyle \int_{-2}^{3} f(x) \;dx$$
\end{oefening}

\section{Oppervlakteberekeningen}

\begin{oefening}
Bepaal de oppervlakte tussen
\begin{enumerate}[(a)]
\itemsep1em
  \item $\displaystyle y=x^2,\quad y=0,\quad x=0, \quad x=2$
  \item $\displaystyle y=x^2-6x,\quad y=0,\quad x=2, \quad x=4$
  \item $\displaystyle y=\sqrt{x},\quad y=0,\quad x=0, \quad x=9$
  \item $\displaystyle y=\sqrt[3]{x},\quad y=0,\quad x=0, \quad x=8$
  \item $\displaystyle y=\dfrac{x-1}{x},\quad y=0,\quad x=-3, \quad x=-1$
  \item $\displaystyle y=\dfrac{x^2-4}{x^2},\quad y=0,\quad x=1, \quad x=6$
\end{enumerate}
\end{oefening}

\begin{oefening}
Bepaal de oppervlakte tussen
\begin{enumerate}[(a)]
\itemsep1em
  \item $\displaystyle y=x(x-3),\quad y=0,\quad x=0, \quad x=5$
  \item $\displaystyle y=\sqrt{x},\quad y=\sqrt{x+1},\quad x=0, \quad x=4$
\end{enumerate}
\end{oefening}


\begin{oefening}
Bepaal de oppervlakte tussen de gegeven kromme en de $x$-as
\begin{multicols}{2}
\begin{enumerate}[(a)]
\itemsep1em
  \item $\displaystyle y=4x-x^2$
  \item $\displaystyle y=4x-x^3$
  \item $\displaystyle y=-x^3+5x^2-7x+3$
  \item $\displaystyle y=4x^4-8x^3-12x^2+16x+16$
\end{enumerate}
\end{multicols}
\end{oefening}

\begin{oefening}
Bepaal de oppervlakte tussen $f$ en $g$
\begin{enumerate}[(a)]
\itemsep1em
  \item $f(x)=x^2$ en $g(x)=2x$
  \item $f(x)=x$ en $g(x)=2x-x^2$
\end{enumerate}
\end{oefening}

\begin{oefening}
Bepaal de oppervlakte tussen
\begin{enumerate}[(a)]
\itemsep1em
  \item $x=y^3-26y+10$ en $x=40-6y^2-y^3$
  \item $x=|y|$ en $x=1-|y|$
  \item $y=|x|$ en $y=x^2-6$
  \item $y=\sqrt{x}$ en $y=x^2$
\end{enumerate}
\end{oefening}

\pagebreak
\section{Onbepaalde integraal}

\begin{oefening} {\scriptsize \em IJkingsproef industrieel ingenieur}\\
Bereken
$$\int \dfrac{(1+x)^2}{\sqrt{x}}\;dx$$
\begin{enumerate}[(A)]
\itemsep.5em
  \item $x^{5/2}+x^{3/2}+x^{1/2} + C$
  \item $\dfrac{(x+1)^3}{3\sqrt{x}} + C$
  \item $\dfrac{2}{5}x^{5/2}+\dfrac{2}{3}x^{3/2}+2x^{1/2} + C$
  \item $\dfrac{2}{5}x^{5/2}+\dfrac{4}{3}x^{3/2}+2x^{1/2} + C$
\end{enumerate}
\end{oefening}

\pagebreak
\section{Integratiemethoden}

\todo{Dit staat nog niet juist}

\begin{oefening}{\scriptsize Bron: Examencommissie Toelatingsexamen Arts en Tandarts, Juli 2016, id: 12190}\\
De afgeleide van een functie $f$, gedefinieerd op $]0,+\infty[$, is gegeven door $f'(x)=\ln x$. Bovendien is $f(e)=e^2$. Dan is $f(e^2)$ gelijk aan \hfill(geen giscorrectie)
\begin{enumerate}[(A)]
  \itemsep.5em
  \item $e^2$
  \item $2e^2$
  \item $2+e^2$
  \item $e^4$
\end{enumerate}
\end{oefening}

\pagebreak
\section{Toepassingen op integralen}


%\newpage
\end{document}
