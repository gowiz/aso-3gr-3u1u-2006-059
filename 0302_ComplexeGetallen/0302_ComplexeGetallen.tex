\documentclass[12pt,twoside]{article}

\input{../gowiz.tex}

\usepackage{versions}
\excludeversion{theorie}
% \includeversion{theorie}

\begin{document}

\pagestyle{fancy}
\lhead{}
\rhead{Oefeningen Complexe Getallen}

\begin{theorie}

  \thispagestyle{empty}
  \begin{center}
    \begin{mdframed}
      \centering
      \fontsize{40}{60}\selectfont Goniometrische Getallen
    \end{mdframed}
    \vfill
    \vfill
  \end{center}

  \subsection*{Doelstellingen}
  \vspace*{-0.8cm}
  {\singlespacing
    Je \hfill  {\scriptsize(LP2006-059, LI1.9, ET32,14,31)}
    \begin{itemize}
      \itemsep-0.2em
    \item kent het begrip complex getal
    \item kan complexe getallen optellen, aftrekken, vermenigvuldigen en delen
    \item kan de $k$-de macht van een complex getal berekenen
    \item kan algebraïsch vierkantswortels uit een complex getal berekenen
    \item kan vergelijkingen van de tweede graad met reële coëfficiënten oplossen in $\mathbb{C}$
    \item kan complexe getallen voorstellen in het vlak van Gauss
    \item kent de goniometrische gedaante van een complex getal
    \item kan product, quotiënt en macht berekenen van complexe
      getallen in goniometrische gedaante
    \item kent de formule van de Moivre
    \item kan goniometrisch de $n$-de wortels berekenen uit een complex getal in goniometrische gedaante
    \end{itemize}}

  \thispagestyle{empty}
  \mbox{}
  \newpage
  \clearpage
  \thispagestyle{empty}
  % \mbox{}
  \tableofcontents
  \newpage
  \clearpage
  \pagenumbering{arabic}

  \fancyhead[RO,LE]{Complexe Getallen}
  \fancyhead[RE,LO]{}

\end{theorie}

\section{Definitie}

\begin{oefening}
  Maak gebruik van een venndiagram om de volgende getallen in de juiste getalverzameling te plaatsten:
  $$-3\qquad\pi\qquad\dfrac{3}{4}\qquad2+3i\qquad10^{10}\qquad-\sqrt{9}$$
\end{oefening}

\begin{oefening}
  Als $i^2=-1$, aan wat is dan $i^4$ gelijk?
\end{oefening}

\begin{oefening}
  Bepaal van de volgende getallen het reëel en het imaginair deel:
  $$2+3i\qquad -i+1\qquad 0 \qquad 2i$$
\end{oefening}

\pagebreak
\section{Basisbewerkingen met complexe getallen}

\begin{oefening}
  Gegeven het complex getal $a+bi$, bepaal het neutraal element en het symmetrisch element in $\mathbb{C}, +$.
\end{oefening}

\begin{oefening}*
  Gegeven het complex getal $a+bi$, bepaal het neutraal element en het symmetrisch element in $\mathbb{C}_0, \cdot$.
\end{oefening}


\begin{oefening}
  Bereken
  \begin{enumerate}[(a)]
  \item $\displaystyle (2-i)+(3+4i)$
  \item $\displaystyle (4i-2)-(2+5i)$
  \item $\displaystyle (7+6i)+2(1+i)$
  \item $\displaystyle (2-3i)\cdot (1-i)$
  \item $\displaystyle 4i(8-2i)$
  \item $\displaystyle (2+i)(-2+i)$
  \item $\displaystyle (1-i)^2$
  \item $\displaystyle (2+3i)^2$
  \item $\displaystyle (2+i)^3$
  \end{enumerate}
\end{oefening}

\begin{oefening}
  Bereken
  \begin{enumerate}[(a)]
    \itemsep1em
  \item $\displaystyle \dfrac{2-4i}{1+i}$
  \item $\displaystyle \dfrac{7+4i}{i}$
  \item $\displaystyle \dfrac{3-2i}{3+2i}$
  \item $\displaystyle \dfrac{-1}{5+2i}$
  \item $\displaystyle \dfrac{1}{i}+1$
  \item $\displaystyle \dfrac{1-3i}{2+3i}$
  \item $\displaystyle \left(\dfrac{1+i\sqrt{2}}{1-i\sqrt{2}}\right)^2+\left(\dfrac{1-i\sqrt{2}}{1+i\sqrt{2}}\right)^2$
  \end{enumerate}
\end{oefening}

\begin{oefening}
  Geef het reëel deel en het imaginair deel van $z=\frac{i-4}{2i-3}$.
\end{oefening}

\section{Machtsverheffing}

\begin{oefening}
  Gegeven de complexe getallen:
  $$z_1=3-2i \qquad z_2=-1-i \qquad z_3=4+5i$$
  Bereken
  \begin{multicols}{2}
    \begin{enumerate}[(a)]
      \itemsep 1em
    \item $\displaystyle z_1+z_2-z_3$
    \item $\displaystyle z_1\cdot z_2$
    \item $\displaystyle \dfrac{z_1}{z_2}$
    \item $\displaystyle \bar{z_2}$
    \item $\displaystyle \dfrac{1}{\bar{z_1}}$
    \item $\displaystyle z_3\cdot \bar{z_3}$
    \item $\displaystyle z_2^2$
    \end{enumerate}
  \end{multicols}
\end{oefening}

\begin{oefening}
  Bereken
  \begin{enumerate}[(a)]
    \itemsep1em
  \item $\displaystyle \left(6+2i\right)^{-2}$
  \item $\displaystyle \left(1+2i\right)^4$
  \item $\displaystyle \dfrac{i^{35}+i^{40}}{1+i^{21}}$
  \item $\displaystyle \dfrac{(1+i)^2+(1-i)^2}{(1+i)^2-(1-i)^2}$
  \item $\displaystyle \left(\dfrac{1+i}{1-i}\right)^{30}+\left(\dfrac{1-i}{1+i}\right)^{30}$
  \end{enumerate}
\end{oefening}

\begin{oefening}
  Vereenvoudig
  \begin{enumerate}[(a)]
    \itemsep 1em
  \item $\frac{1+i}{1-i}-(1+2i)(2+2i)+\frac{3-i}{1+i}$;
  \item $2i(i-1)+\left(\overline{\sqrt(3)+i}\right)^3+(1+i)\overline{(1+i)}.$
  \end{enumerate}
\end{oefening}

\begin{oefening}
  We definiëren de {\bf positieve reële getallen} als $\mathbb{R}^+$. Dit zijn dus alle reële getallen met een positief teken, inclusief nul. We definiëren analoog de {\bf negatieve reële getallen} als $\mathbb{R}^-$. Dit zijn dus alle reële getallen met een negatief teken, inclusief nul.

  De positieve reële getallen en de negatieve reële getallen hebben dus het getal nul gemeenschappelijk:
  $$\mathbb{R}^+\cap\mathbb{R}^-=\{0\}$$

  {\bf Niet-positieve reële getallen} zijn dan alle reële getallen zonder de positieve reële getallen, {\bf niet-negatieve reële getallen} zijn dan alle reële getallen zonder de negatieve reële getallen.

  Als we een complex getal verschillend van nul vermenigvuldigen met zijn complex toegevoegde, dan krijgen we altijd een
  \begin{enumerate}[(A)]
  \item positief reëel getal.
  \item negatief reëel getal.
  \item niet-positief reëel getal.
  \item niet-negatief reëel getal.
  \end{enumerate}
\end{oefening}

\pagebreak
\section{Vierkantswortel van een complex getal}

\begin{oefening}
  Bereken alle vierkantswortels van
  \begin{enumerate}[(a)]
    \itemsep 1em
  \item $-9$
  \item $2i$
  \item $-3-4i$
  \item $-i$
  \end{enumerate}

\end{oefening}


\end{document}
